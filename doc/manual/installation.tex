\subsection{Obtaining \dftfe{}}
The development version of \dftfe{} can be downloaded by executing the following command
\begin{verbatim}
   git clone https://userid@bitbucket.org/dftfedevelopers/dftfe.git
\end{verbatim}
For internal group testing, use the \verb|internalGroupTesting| branch.
\begin{verbatim}
   git checkout internalGroupTesting
\end{verbatim}

\subsection{Compiling required external libraries}
Not required for internal group testing if trying to run \dftfe{} on flux.

\subsection{Compiling \dftfe{}}
Following are the steps to compile DFT-FE (only for flux)
\begin{enumerate}
\item   \begin{verbatim}
cd dftfe
\end{verbatim}
\item \begin{verbatim}
git checkout internalGroupTesting
\end{verbatim}
\item  If compiling on flux, do
\begin{verbatim}
module purge
module load cmake
module load intel/18.0.1
module load openmpi/3.0.0/intel/18.0.1
\end{verbatim}
and run
\begin{verbatim}
./setup.sh
\end{verbatim} 
\end{enumerate}
, which by default compiles in the release mode (the fast version). If \dftfe{} compiled successfully, the following executables will be created in the \verb|build| directory:
\begin{verbatim}
/build/release/real/main
/build/release/complex/main
\end{verbatim}
One could also set
\begin{verbatim}
optimizedFlag=1
\end{verbatim}
inside \verb|setup.sh|, and run
\begin{verbatim}
./setup.sh
\end{verbatim}
, which will create debug mode executables (useful for debugging purposes but much slower than the release mode executable) in the folders
\begin{verbatim}
/build/debug/real/main
/build/debug/complex/main
\end{verbatim}
 
