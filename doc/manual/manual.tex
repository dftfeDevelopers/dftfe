\documentclass{article}
\usepackage[pdftex]{graphicx,color}
\usepackage[T1]{fontenc}
\usepackage{lmodern}
\usepackage{amsmath}
\usepackage{amsfonts}
\usepackage{subcaption}
\usepackage{textpos}

% use a larger page size; otherwise, it is difficult to have complete
% code listings and output on a single page
\usepackage{fullpage}

% have an index. we use the imakeidx' replacement of the 'multind' package so
% that we can have an index of all run-time parameters separate from other
% items (if we ever wanted one)
\usepackage{imakeidx}
\makeindex[name=prmindex, title=Index of run-time parameter entries]
\makeindex[name=prmindexfull, title=Index of run-time parameters with section names]

% be able to use \note environments with a box around the text
\usepackage{fancybox}
\newcommand{\note}[1]{
{\parindent0pt
  \begin{center}
    \shadowbox{
      \begin{minipage}[c]{0.9\linewidth}
        \textbf{Note:} #1
      \end{minipage}
    }
  \end{center}
}}

% use the listings package for code snippets. define keywords for prm files
% and for gnuplot
\usepackage{listings}
\lstset{
  language=C++,
  showstringspaces=false,
  basicstyle=\small\ttfamily,
  columns=fullflexible,
  keepspaces=true,
  frame=single,
  breaklines=true,
  postbreak=\raisebox{0ex}[0ex][0ex]{\hspace{5em}\ensuremath{\color{red}\hookrightarrow\space}}
}
\lstdefinelanguage{prmfile}{morekeywords={set,subsection,end},
                            morecomment=[l]{\#},escapeinside={\%\%}{\%},}
\lstdefinelanguage{gnuplot}{morekeywords={plot,using,title,with,set,replot},
                            morecomment=[l]{\#},}


% use the hyperref package; set the base for relative links to
% the top-level \dftfe directory so that we can link to
% files in the \dftfe tree without having to specify the
% location relative to the directory where the pdf actually
% resides
\usepackage[colorlinks,linkcolor=blue,urlcolor=blue,citecolor=blue,baseurl=../]{hyperref}

\newcommand{\dealii}{{\textsc{deal.II}}}
\newcommand{\pfrst}{{\normalfont\textsc{p4est}}}
\newcommand{\trilinos}{{\textsc{Trilinos}}}
\newcommand{\petsc}{{\textsc{PETSc}}}
\newcommand{\dftfe}{\textsc{DFT-FE}}

\begin{document}

%%%%%%%%%%%%%%%%%%%%%%%%%%%%%%
%%% START OF DFT-FE MANUAL COVER TEMPLATE %%%
%%%%%%%%%%%%%%%%%%%%%%%%%%%%%%
% This should be pasted at the start of manuals and appropriate strings entered at locations indicated with FILL.
% Be sure the TeX file includes the following packages.
% \usepackage{graphicx}
% \usepackage{times}
% \usepackage{textpos}

\definecolor{dark_grey}{gray}{0.3}
\definecolor{dftfe_blue}{rgb}{0.0,0.39,0.76}

%LINE 1%
{
\renewcommand{\familydefault}{\sfdefault}

\pagenumbering{gobble}
%\begin{center}
%\resizebox{\textwidth}{!}{\textcolor{dark_grey}{\fontfamily{\sfdefault}\selectfont
%	\href{http://www-personal.umich.edu/~vikramg/}{COMPUTATIONAL MATERIAL PHYSICS GROUP}
%}}
%\vspace{0.05em}
%\hrule

%LINE 2%
%\color{dark_grey}
%\rule{\textwidth}{2pt}

%LINE 3%
%\color{dark_grey}
% FILL: additional organizations
% e.g.: {\Large Organization 1\\Organization 2}
%{\Large }
%\end{center}

%COLOR AND CODENAME BLOCK%
\begin{center}
\resizebox{\textwidth}{!}{\colorbox
% FILL: color of code name text box
% e.g. blue
{dftfe_blue}{\fontfamily{\rmdefault}\selectfont \textcolor{yellow} {
% FILL: name of the code
% You may want to add \hspace to both sides of the codename to better center it, such as:
% \newcommand{\codename}{\hspace{0.1in}CodeName\hspace{0.1in}}
\hspace{0.1in}\dftfe{}\hspace{0.1in}
}}}
\\[12pt]
{\Large Density Functional Theory calculations with Finite-Elements}
\end{center}

%MAIN PICTURE%
\begin{textblock*}{0in}(0.5in,0.3in)
% FILL: image height
% e.g. height=6.5in
\begin{center}
\vspace{1em}
\includegraphics[scale=0.35]{N2.png}
% FILL: image file name
% e.g. cover_image.png
%{contour_5x5x5.pdf}
%{SiCTriplet0000.png}
\hspace{5em}
\end{center}
\end{textblock*}

%USER MANUAL%
\color{dark_grey}
\vspace{1.0em}
\hfill{\Huge \fontfamily{\sfdefault}\selectfont User Manual \\
\raggedleft \huge \fontfamily{\sfdefault}\selectfont Version
% keep the following line as is so that we can replace this using a script:
1.0 %VERSION-INFO%
\\\large(generated \today)\\
\vspace{1.5em}
{\Large Sambit Das\,\\Vikram Gavini\,\\Phani Motamarri\\}
\vspace{1.0em}
\large
\noindent with contributions by: \\
    {\Large Krishnendu Ghosh\\}
\vspace{1.0em}
}
%WEBSITE%
\null
\vspace{17em}

{\noindent
{\fontfamily{\sfdefault}\selectfont \href{https://sites.google.com/umich.edu/dftfe}{website of dftfe}}
}

%\begin{textblock*}{0in}(5.25in,-0.8in)
%\includegraphics[height=0.8in]{CoE-vert.png}
%\end{textblock*}

%LINE%
{\noindent
\color{dark_grey}
\rule{\textwidth}{2pt}
}

}
Copyright (c) 2017-2021 The Regents of the University of Michigan and \hyperref[sec:authors]{DFT-FE authors}.
\pagebreak
\pagenumbering{arabic}

%%%%%%%%%%%%%%%%%%%%%%%%%%%%%%
%%%   END OF DFT-FE MANUAL COVER TEMPLATE    %%%
%%%%%%%%%%%%%%%%%%%%%%%%%%%%%%

\pagebreak

\tableofcontents

\pagebreak

\section{Introduction}
\label{sec:intro}
\dftfe{} is a C++ code for materials modeling from first principles using Kohn-Sham density functional theory.
It is based on adaptive finite-element discretization that handles all-electron and pseudopotential calculations in the 
same framework, and incorporates scalable and efficient solvers for the solution of the Kohn-Sham equations. Importantly, \dftfe{} 
can handle general geometries and boundary conditions, including periodic, semi-periodic and non-periodic systems. \dftfe{} code 
builds on top of the deal.II library for everything that has to do with finite elements, geometries, meshes, etc., and, through 
deal.II on p4est for parallel adaptive mesh handling.

\subsection{Authors}
\label{sec:authors}
\dftfe{} is developed and maintained by the \href{http://www-personal.umich.edu/~vikramg/}{Computational Materials Physics
group at University of Michigan}, and the \href{https://sites.google.com/view/matrix-lab}{MATRIX lab, Indian Institute of Science}. The code is maintained by a group of principal developers, 
who manage the architecture of the code and the core functionalities. Developers with
significant contributions to core functionalities and code architecture in the past who are 
no longer active principal developers, are listed under principal developers emeriti. 
A subset of the principal developers and mentors are administrators. Finally, all contributors who have
contributed to major parts of the DFT-FE code or sent important fixes and enhancements are listed
under contributors. All the underlying lists are in alphabetical order. 

\paragraph{Principal developers}
\begin{itemize}
	\item Sambit Das (University of Michigan, USA).
	\item Phani Motamarri (Indian Institute of Science, Bangalore, India).
\end{itemize}

\paragraph{Principal developers emeriti}
\begin{itemize}
	\item Krishnendu Ghosh (University of Michigan, USA).
	\item Shiva Rudraraju (University of Wisconsin Madison, USA).	
\end{itemize}

\paragraph{Contributors}
\begin{itemize}
	\item Sambit Das (University of Michigan, USA).
	\item Vishal Subramanian (University of Michigan, USA).
	\item Bikash Kanungo (University of Michigan, USA).
	\item Denis Davydov (University of Erlangen-Nuremberg, Germany).
	\item Krishnendu Ghosh (Intel Corp., USA).
	\item Phani Motamarri (Indian Institute of Science, Bangalore, India).
	\item Shiva Rudraraju (University of Wisconsin Madison, USA).
	\item Shukan Parekh (University of Michigan, USA). 	
	\item David Rogers (Oak Ridge National Laboratory, USA).
\end{itemize}

\paragraph{Mentors}
\begin{itemize}
	\item Vikram Gavini (University of Michigan, USA).
\end{itemize}

\subsection{Acknowledgments}
The development of \dftfe{} open source code relating to pseudopotential calculations has been funded in part by the Department 
of Energy PRISMS Software Center at University of Michigan, and the Toyota Research Institute. The development of \dftfe{} open 
source code relating to all-electron calculations has been funded by the Department of Energy Basic Energy Sciences. The methods 
and algorithms that have been implemented in \dftfe{} are outputs from research activities over many years that have been 
supported by the Army Research Office, Air Force Office of Scientific Research, Department of Energy Basic Energy Sciences, 
National Science Foundation and Toyota Research Institute. 

\subsection{Referencing \dftfe{}}
Please refer to \href{https://sites.google.com/umich.edu/dftfe/referencing}{referencing  \dftfe{}} to properly cite the use of 
\dftfe{} in your scientific work. 


\section{Useful background information}
\label{sec:background}
\input{background}

\section{Installation}
\label{sec:installation}
All the underlying installation instructions assume a Linux operating system. We assume standard tools and libraries like CMake, compilers- (C, C++ and Fortran), CUDA (in case of GPU architectures), MPI, and math(BLAS-LAPACK) libraries are pre-installed. Most high-performance computers would have the latest version of these libraries in the default environment. However, in many cases you would have to use \href{http://modules.sourceforge.net/}{Environment Modules} to set the correct environment variables for the above and compilation tools like \href{http://www.cmake.org/}{CMake}. \emph{For example, on one of the high-performance computers (UMICH Greatlakes) we develop and test the \dftfe{} code, we can use the following commands to set the desired environment variables}
\begin{verbatim}
$ module load cmake/3.18.2
$ module load gcc/8.2.0
$ module load openmpi/3.1.4
$ module load mkl/2018.0.4
$ module load cuda/11.0.2 (if installing for GPUs)
\end{verbatim}
%In the above mpilibrary denotes the MPI library you are using in your system(for eg: openmpi, mpich or intel-mpi). 
Note the above are shown only as an example. We strongly recommend using the latest stable version of compilers-(C, C++ and Fortran), CUDA, MPI and math libraries available on your high-performance computer. DFT-FE's installation requires also the following minimum versions of the above compilers and libraries:
\begin{itemize}
    \item CMake 3.17.0
    \item GCC 8.2.0
    \item CUDA 11.0.2 (if installing for GPUs)
\end{itemize}
\textcolor{red}{\bf We currently do not support Intel compilers due to a compilation issue of the deal.II library. Please use GNU compilers only.} Further, if version of CMake greater than 3.17.0 is not available on your machine please install latest version from here \href{http://www.cmake.org/}{CMake} or use pre-installed binaries most appropriate for your machine.

\subsection{Compiling and installing external libraries}
\dftfe{} is primarily based on the open source finite element library \href{http://www.dealii.org/}{deal.II}, through which external dependencies
on \href{http://p4est.org/}{p4est}, \href{http://www.netlib.org/scalapack/}{ScaLAPACK} and BLAS-LAPACK are set. The other required external libraries, which are
not interfaced via deal.II are \href{http://www.alglib.net/}{ALGLIB}, \href{http://www.tddft.org/programs/libxc/}{Libxc}, \href{https://atztogo.github.io/spglib/}{spglib}, \href{http://www.xmlsoft.org/}{Libxml2} and \href{https://elpa.mpcdf.mpg.de/}{ELPA}. DFT-FE also optionally links to \href{https://www.mcs.anl.gov/petsc/}{PETSc}, \href{http://slepc.upv.es/}{SLEPc} (via deali.II) and to \href{https://developer.nvidia.com/nccl}{nccl} (for GPU compilation). The optional dependencies of PETSc and SLEPc are only required for all-electron calculations using DFT-FE, which uses the more stable Gram-Schmidt orthogonalization routine instead of the default Cholesky-Gram-Schmidt orthognalization. For pseudopotential calculations, PETSc and SLEPc dependencies are not required as the default Cholesky-Gram-Schmidt orthogonalization is very robust. Below, we give brief installation instructions for each of the above libraries.
\subsubsection{Instructions for dependencies: ALGLIB, Libxc, spglib, Libxml2, ScaLAPACK, ELPA, p4est and nccl (nccl is optional)}
\begin{enumerate}
	\item   {\bf ALGLIB}: Used by \dftfe{} for spline fitting for various radial data. Download the current release of the Alglib free C++ edition from \url{http://www.alglib.net/download.php}. After downloading and unpacking, go to \verb|cpp/src|, and create a shared library using a C++ compiler. For example, using GCC compiler do
\begin{verbatim}
$ g++ -c -fPIC -O2 *.cpp
$ g++ *.o -shared -o libAlglib.so
\end{verbatim}
\item {\bf Libxc}: Used by \dftfe{} for exchange-correlation functionals. Download the current release from \url{http://www.tddft.org/programs/libxc/download/}, and do 
\begin{verbatim}
$ ./configure --prefix=libxc_install_dir_path
              CC=c_compiler CXX=c++_compiler FC=fortran_compiler
	       CFLAGS="-O2 -fPIC" FCFLAGS="-O2 -fPIC" CXXFLAGS="-O2 -fPIC"
     
$ make
$ make install
\end{verbatim}
Do not forget to replace \verb|libxc_install_dir_path| by some appropriate path on your file system and make sure that you have write permissions. Also replace \verb|c_compiler, c++_compiler| and \verb|fortran_compiler| with compilers on your system.

\item {\bf spglib}: Used by \dftfe{} to find crystal symmetries. To install spglib, first obtain the development version of spglib from their github repository by
\begin{verbatim}
$ git clone https://github.com/atztogo/spglib.git	
\end{verbatim}	
and next follow the ``Compiling using cmake'' installation procedure described in \url{https://atztogo.github.io/spglib/install.html}.   	
We recommend using the ccmake gui interface for the installation and also use appropriate compiler for \verb|CMAKE_C_COMPILER|.

\item {\bf Libxml2}: Libxml2 is used by \dftfe{} to read \verb|.xml| files. Most likely, Libxml2 might be already installed in the high-performance computer you are working with. It is usually installed in the default locations like \verb|/usr/lib64| (library path which contains \verb|.so| file for Libxml2, something like \verb|libxml2.so.2|) and \verb|/usr/include/libxml2| (include path). 

Libxml2 can also be installed by doing (Do not use these instructions if you have already have Libxml2 on your system)
\begin{verbatim}
$ git clone https://gitlab.gnome.org/GNOME/libxml2.git
$ ./autogen.sh --prefix=Libxml_install_dir_path
$ make
$ make install 
\end{verbatim}
There might be errors complaining that it can not create regular file libxml2.py in /usr/lib/python2.7/site-packages, but that should not matter.

\item {\bf ScaLAPACK}: ScaLAPACK library is used by DFT-FE via deal.II for its parallel linear algebra routines involving dense matrices, as well being a dependency for ELPA. \textcolor{red}{\bf If Intel MKL math library is available, please skip this step, as the ScaLAPACK libraries therein can be used directly.} If Intel MKL math library is not available, Netlib ScaLAPACK \url{http://www.netlib.org/scalapack/} needs to be installed using the instructions below. Download the current release version (2.1.0) from \url{http://www.netlib.org/scalapack/#\_software}, and build a shared library (use \verb|BUILD_SHARED_LIBS=ON|, \verb|BUILD_STATIC_LIBS=OFF| and \verb|BUILD_TESTING=OFF|) installation of ScaLAPACK using cmake. We recommend using the ccmake gui interface for the installation.  Further, use the appropriate compilers for \verb|CMAKE_C_COMPILER| and \verb|CMAKE_FORTRAN_COMPILER|, and also use \verb|-fPIC| flag for \verb|CMAKE_C_FLAGS| and \verb|CMAKE_Fortran_FLAGS|. For best performance, ScaLAPACK must be linked to optimized BLAS-LAPACK libraries by using\\ \verb|USE_OPTIMIZED_LAPACK_BLAS=ON|, and providing external paths to BLAS-LAPACK libraries (MKL, OpenBlas, ESSL etc.) during the cmake configuration. 
%Alternatively one can also use the python based installer~\url{http://www.netlib.org/scalapack/scalapack_installer.tgz} for Linux.

\item {\bf ELPA}: ELPA library is used by DFT-FE for its parallel linear algebra routines involving dense matrices. ELPA requires the ScaLAPACK library (see above) as a dependency. Download version elpa-2020.05.002 (this version has been thoroughly tested) from \url{https://elpa.mpcdf.mpg.de/software/} and follow the installation instructions in there. Example of ELPA installation on UMICH Greatlakes supercomputer with GNU compiler, Open MPI library, and Intel MKL math library:
\begin{verbatim}
$ cd elpaDir
$ mkdir build
$ cd build
$ ../configure --enable-openmp FC=mpif90 CC=mpicc CXX=mpicxx 
FCFLAGS="-fopenmp -O2 -march=native" CFLAGS="-fopenmp -O2 -march=native" 
CXXFLAGS="-fopenmp -O2 -march=native" --prefix=elpa_install_path
SCALAPACK_LDFLAGS=" -L${MKLROOT}/lib/intel64 -lmkl_scalapack_lp64
-Wl,--no-as-needed -lmkl_intel_lp64
-lmkl_gnu_thread -lmkl_core -lmkl_blacs_openmpi_lp64 -lgomp -lpthread -lm -ldl" 
SCALAPACK_FCFLAGS="-L${MKLROOT}/lib/intel64 -lmkl_scalapack_lp64
-Wl,--no-as-needed -lmkl_gf_lp64 -lmkl_gnu_thread -lmkl_core
-lmkl_blacs_openmpi_lp64 -lgomp -lpthread -lm -ldl"
$ make -j 4
$ make install
\end{verbatim}
The MKL paths and linker flags were obtained with the help of \href{https://software.intel.com/en-us/articles/intel-mkl-link-line-advisor}{Intel MKL Link Line Advisor} for GNU Fortran compiler (Note the usage of \verb|-lmkl_gf_lp64| flag above).\\ 

If the machine of interest has NVIDIA GPUs and CUDA library, ELPA can take advantage of GPUs. For example on OLCF Summit GPU nodes, we use the following configure line
\begin{verbatim}
../configure FC=mpif90 CC=mpicc FCFLAGS="-O2
-fPIC -mcpu=power9 -mvsx -maltivec" CFLAGS="-O2
-fPIC -mcpu=power9 -mvsx -maltivec"
--enable-nvidia-gpu
--with-NVIDIA-GPU-compute-capability="sm_70"
--with-cuda-path="$OLCF_CUDA_ROOT"
--with-cuda-sdk-path="$OLCF_CUDA_ROOT"
--prefix=elpa_install_path
LDFLAGS="-L$netlib_scalapack_installation_path/lib -lscalapack  
-L/$OLCF_ESSL_ROOT/lib64 -lessl
-L/$OLCF_NETLIB_LAPACK_ROOT/lib64 -llapack
-L/$OLCF_OPENBLAS_ROOT/lib -lopenblas"
--disable-sse --disable-sse-assembly
--disable-avx --disable-avx2 --disable-avx512
--enable-c-tests=no
\end{verbatim}
Note the use of LDFLAGS instead of SCALAPACK\_LDFLAGS and SCALAPACK\_FCFLAGS, since we are using netlib ScaLAPACK instead of Intel MKL ScaLAPACK in the above. Also note use of
\begin{verbatim}
    --disable-sse --disable-sse-assembly --disable-avx --disable-avx2 --disable-avx512
\end{verbatim} 
above. \textcolor{red}{\bf Some or all of these options may be required for systems without Intel CPUs such as IBM Power and AMD Epyc processors depending on what they support.}

\item   {\bf p4est}: This library is used by deal.II to create and distribute finite-element meshes across multiple processors. Download the v2.2 tarball of p4est from \url{http://www.p4est.org/}. Next download the \verb|p4est-setup.sh| script from \url{https://dealii.org/developer/external-libs/p4est.html}. Use the script to automatically compile and install a debug and optimized version of p4est by doing
\begin{verbatim}
$ chmod u+x p4est-setup.sh
$ ./p4est-setup.sh p4est-x-y-z.tar.gz p4est_install_dir_path
\end{verbatim}

%Also download the script from \url{https://github.com/dftfeDevelopers/dftfe/raw/manual/p4est-setup.sh} if using Intel compilers, or from \url{https://dealii.org/developer/external-libs/p4est.html} if using GCC compilers. Use the script to automatically compile and install a debug and optimized version of p4est by doing

\item {\bf nccl (optional)}: nccl is an optional dependency for DFT-FE for optimal GPU Direct MPI collective communication calls. This library is recommended for running very large system sizes (greater than 20,000 electrons) on GPUs using DFT-FE. Caution: nccl library requires appropriate hardware support for GPU Direct MPI communication and CUDA Aware MPI library. To install nccl, clone development version \url{https://github.com/NVIDIA/nccl} and follow installation instructions therein.

\end{enumerate}


\subsubsection{Instructions for deal.II}
Assuming the above dependencies (p4est and ScaLAPACK) are installed, we now briefly discuss the steps to compile and install deal.II library linked with the above dependencies. You need to install two variants of the deal.II library-- one variant linked with real scalar type PETSc and SLEPc installations, and the other variant linked with complex scalar type PETSc and SLEPc installations. 

\begin{enumerate}

\item Obtain the customized version of deal.II library via
\begin{verbatim}
$ git clone -b dealiiCustomizedCUDARelease https://github.com/dftfeDevelopers/dealii.git
\end{verbatim}

\item In addition to requiring C, C++ and Fortran compilers, MPI library, and CMake, deal.II additionaly requires BOOST library. If not found externally, cmake will resort to the bundled BOOST that comes along with deal.II. Based on our experience, we recommend to use the deal.II's bundled boost (enforced by unsetting/unloading external BOOST library environment paths) to avoid compilation issues. Further if installing on NVIDIA GPUs, deal.II requires CUDA (minimum version 11.0.2).

\item
\begin{verbatim}
$ mkdir build
$ cd build
$ cmake -DCMAKE_INSTALL_PREFIX=dealii_install_dir_path
        otherCmakeOptions ../deal.II
$ make -j 8        
$ make install
\end{verbatim}
{\bf ``otherCmakeOptions'' include} the following options for CPU installation:
\begin{verbatim}
-DCMAKE_C_COMPILER=c_compiler
-DCMAKE_CXX_COMPILER=cxx_compiler
-DCMAKE_Fortran_COMPILER=fortran_compiler
-DMPI_C_COMPILER=mpi_c_compiler_wrapper
-DMPI_CXX_COMPILER=mpi_cxx_compiler_wrapper
-DMPI_Fortran_COMPILER=mpi_fortran_compiler_wrapper
-DCMAKE_CXX_FLAGS=cxx_flags
-DCMAKE_C_FLAGS=c_flags
-DDEAL_II_WITH_MPI=ON -DDEAL_II_WITH_64BIT_INDICES=ON
-DDEAL_II_WITH_P4EST=ON -DP4EST_DIR=p4est_install_dir_path
-DDEAL_II_WITH_LAPACK=ON
-DLAPACK_DIR=lapack_dir_paths (both BLAS and LAPACK directory paths)
-DLAPACK_FOUND=true
-DLAPACK_LIBRARIES=lapack_lib_paths (both BLAS and LAPACK library paths)
-DDEAL_II_WITH_SCALAPACK=ON
-DSCALAPACK_DIR=scalapack_dir_path (only required if linking to netlib ScaLAPACK)
-DSCALAPACK_LIBRARIES=scalapack_lib_path
-DDEAL_II_WITH_TBB=OFF
-DDEAL_II_WITH_TASKFLOW=OFF
-DDEAL_II_COMPONENT_EXAMPLES=OFF
\end{verbatim}
and additional cmake options for GPU installation:
\begin{verbatim}
-DDEAL_II_CUDA_FLAGS=gpu_arch_flag (eg. "-arch=sm_70")
-DDEAL_II_WITH_CUDA=ON 
-DDEAL_II_MPI_WITH_CUDA_SUPPORT=ON
\end{verbatim}


\end{enumerate}	
 For more information about installing deal.II library refer to \url{https://dealii.org/developer/readme.html}. We also provide here an example of deal.II installation, which we did on UMICH Greatlakes supercomputer with GNU compiler, Open MPI library, and Intel MKL math library
\begin{verbatim}
$ mkdir build
$ cd build
$ cmake -DCMAKE_C_COMPILER=gcc 
-DCMAKE_CXX_COMPILER=g++
-DCMAKE_Fortran_COMPILER=gfortran
-DMPI_C_COMPILER=mpicc 
-DMPI_CXX_COMPILER=mpicxx 
-DMPI_Fortran_COMPILER=mpif90
-DCMAKE_CXX_FLAGS="-march=native"
-DCMAKE_C_FLAGS="-march=native"
-DDEAL_II_CXX_FLAGS_RELEASE="-O2"
-DDEAL_II_COMPONENT_EXAMPLES=OFF
-DDEAL_II_WITH_MPI=ON
-DDEAL_II_WITH_64BIT_INDICES=ON
-DDEAL_II_WITH_TBB=OFF
-DDEAL_II_WITH_TASKFLOW=OFF 
-DDEAL_II_WITH_P4EST=ON 
-DP4EST_DIR=p4est_install_path 
-DDEAL_II_WITH_LAPACK=ON -DLAPACK_DIR="${MKLROOT}/lib/intel64"
-DLAPACK_FOUND=true
-DLAPACK_LIBRARIES="-L${MKLROOT}/lib/intel64
-Wl,--no-as-needed -lmkl_intel_lp64
-lmkl_gnu_thread -lmkl_core -lgomp -lpthread -lm
-ldl" -DLAPACK_INCLUDE_DIRS="-I${MKLROOT}/include" 
-DDEAL_II_WITH_SCALAPACK=ON
-DSCALAPACK_LIBRARIES="-L${MKLROOT}/lib/intel64
-lmkl_scalapack_lp64 -Wl,--no-as-needed
-lmkl_intel_lp64 -lmkl_gnu_thread -lmkl_core
-lmkl_blacs_openmpi_lp64 -lgomp -lpthread -lm
-ldl"
-DCMAKE_INSTALL_PREFIX=dealii_install_path ../dealii
$ make -j 8
$ make install
\end{verbatim}
The values for \verb|-DLAPACK_DIR|,\verb|-DLAPACK_LIBRARIES| and \verb|-DLAPACK_LINKER_FLAGS| were obtained with the help of \href{https://software.intel.com/en-us/articles/intel-mkl-link-line-advisor}{Intel MKL Link Line Advisor} for GNU C++ compiler (cf. Fig.~\ref{fig:intelmkl}).\\ 
\begin{figure}[htp]
    \centering
\includegraphics[scale=0.4]{intelmklLineAdvisorExampleNew.png}
    \caption{Example usage of Intel MKL line advisor. Use the options in the ``link line'' generated by the line advisor tool. Please note to change ``intelmpi'' in ``-lmkl\_blacs\_intelmpi\_lp64'' to ``openmpi'' if using openmpi MPI library.}
    \label{fig:intelmkl}
\end{figure}

%Note that in the above procedure one is installing the development version of deal.II library and this version is continuously updated by deal.II developers, which can sometimes lead to installation issues on certain compilers. If you face any issues during the installation procedure of deal.II development version as explained above, you may alternatively obtain the current release version of deal.II by downloading and unpacking the .tar.gz file from \url{https://www.dealii.org/download.html} and following the same procedure as above. If you still face installation issues, and/or if you have any questions about the deal.II installation, please contact the deal.II developers at \href{https://groups.google.com/d/forum/dealii}{deal.II mailing lists}.\\

{\bf Using AVX, AVX-512 instructions in deal.II:}\\
deal.II compilation will automatically try to pick the available vector instructions on the sytem like SSE2, AVX and AVX-512, and generate the following output message during compilation   
\begin{verbatim}
-- Performing Test DEAL_II_HAVE_SSE2
-- Performing Test DEAL_II_HAVE_SSE2 - Success/Failed
-- Performing Test DEAL_II_HAVE_AVX
-- Performing Test DEAL_II_HAVE_AVX - Success/Failed
-- Performing Test DEAL_II_HAVE_AVX512
-- Performing Test DEAL_II_HAVE_AVX512 - Success/Failed
-- Performing Test DEAL_II_HAVE_OPENMP_SIMD
-- Performing Test DEAL_II_HAVE_OPENMP_SIMD - Success/Failed
\end{verbatim}
``Success'', means deal.II was able to use the corresponding vector instructions, and ``Failed'' would mean otherwise. If deal.II is not able to pick an available vector instruction on your high-performance computer, please contact the deal.II developers at \href{https://groups.google.com/d/forum/dealii}{deal.II mailing lists} and/or contact your high-performance computer support for guidance on how to use the correct compiler flags for AVX or AVX-512. 

Ensure that deal.II picks up AVX-512, which is strongly recommended for obtaining maximum performance on the new Intel Xeon Phi (KNL) and Skylake processors, both of which support Intel AVX-512 instructions.


\subsection{Obtaining and Compiling \dftfe{}}\label{sec:dftfeinstall}
Assuming that you have already installed the above external dependencies, next follow the steps below to obtain and compile \dftfe{}.
\begin{enumerate}
\item Obtain the source code of the current release of \dftfe{} with all current patches using \href{https://git-scm.com/}{Git}:
\begin{verbatim}
$ git clone -b release1.0 https://github.com/dftfeDevelopers/dftfe.git
$ cd dftfe
\end{verbatim}
Do \verb|git pull| in the \verb|dftfe| directory any time to obtain new patches that have been added since your \verb|git clone| or last \verb|git pull|.
If you are not familiar with Git, you may download the current release tarball from the \href{https://sites.google.com/umich.edu/dftfe/download}{Downloads} page in our website, but downloading via Git is recommended to avail new patches seamlessly. 


%{\bf Obtaining previous releases:} (Skip this part if you are using the current release version)
%\begin{enumerate}
%\item
%\begin{verbatim}
%$ git clone https://github.com/dftfeDevelopers/dftfe.git 
%$ cd dftfe
%\end{verbatim}

%\item To get the list of all release tags (current and previous releases) do
%\begin{verbatim}
%$ git tag -l
%\end{verbatim}

%\item
%Choose the desired release tag name and do
%\begin{verbatim}
%$ git checkout tags/<tag_name> 
%\end{verbatim}
%\end{enumerate}
%Alternatively, you could download the appropriate tarball from \href{https://github.com/dftfeDevelopers/dftfe/releases}{github-releases}.

\item Set paths to external libraries (deal.II, ALGLIB, Libxc, spglib, Libxml2, and ELPA), C++ compiler, and C++ compiler flags in \verb|setupUser.sh|, which is a script to compile \dftfe~using cmake. nccl library can also be optionally provided in case of GPU compilation. If compiling only for CPUs, set the following to OFF
\begin{verbatim}
withGPU=OFF
withNCCL=OFF
\end{verbatim}
For GPU compilation, \verb|withGPU| must be set to ON.
\item To compile \dftfe{}, first create a build directory anywhere on your machine. Next from inside the build directory do
\begin{verbatim}
$ bash $dftfe_source/setupUser.sh
\end{verbatim} 
Please use the full directory path for \verb|$dftfe_source| above. Also note that sometimes compilation on login node can crash due to insufficient memory. In those cases, we recommend using an interactive job if available on your computing cluster.

\item If compilation is successful, the following executables will be created:
\begin{verbatim}
$dftfe_build_dir/release/real/dftfe
$dftfe_build_dir/release/complex/dftfe
\end{verbatim}

\item
To compile \dftfe{} in debug mode (much slower but useful for debugging), set \verb|build_type=Debug| in \verb|setupUser.sh| and do:
\begin{verbatim}
$ bash $dftfe_source/setupUser.h
\end{verbatim}
which will create the following debug mode executables:
\begin{verbatim}
$dftfe_build_dir/debug/real/dftfe
$dftfe_build_dir/debug/complex/dftfe
\end{verbatim}
\end{enumerate}



\subsection{Optional PETSc, SLEPc, deal.II and \dftfe~ installation instructions for all-electron calculations using DFT-FE}
Users can ignore this section if only interested in pseudopotential calculations. The optional dependencies of PETSc and SLEPc are only required for all-electron calculations using DFT-FE, which uses the more stable Gram-Schmidt orthogonalization routine instead of the default Cholesky-Gram-Schmidt orthognalization. 
    
\begin{enumerate}
\item {\bf PETSc}: PETSc is a parallel linear algebra library. \dftfe{} with PETSc and SLEPc dependencies needs two variants of the PETSc installation---one with real scalar type and the another with complex scalar type. Also both the installation variants must have 64-bit indices and optimized mode enabled during the installation. To install PETSc, first download the current release (3.15.0 or later) tarball from \url{https://www.mcs.anl.gov/petsc/download/index.html}, unpack it, and follow the installation instructions in \url{https://www.mcs.anl.gov/petsc/documentation/installation.html}. 
	
Below, we show an example installation for the real scalar type variant. 
This example should be used only as a reference.
\begin{verbatim}
$ ./configure --prefix=petscReal_install_dir_path --with-debugging=no 
              --with-64-bit-indices=true --with-cc=c_compiler
              --with-cxx=c++_compiler --with-fc=fortran_compiler
              --with-blas-lapack-lib=(optimized BLAS-LAPACK library path) 
              CFLAGS=c_compilter_flags CXXFLAGS=c++_compiler_flags
	              FFLAGS=fortran_compiler_flags

$ make PETSC_DIR=prompted by PETSc 
       PETSC_ARCH=prompted by PETSc

$ make PETSC_DIR=prompted by PETSc 
       PETSC_ARCH=prompted by PETSc
       install
\end{verbatim}
For the complex installation variant, unpack a fresh PETSc directory (required) from the tarball and repeat the above steps with the only changes being adding  \verb|--with-scalar-type=complex| and \verb|--with-fortran-kernels=true| to the configuration step (\verb|./configure|) as well as providing a new installation path to \verb|--prefix|. Below we provide example configure lines for real and complex versions on UMICH Greatlakes supercomputer with GNU compiler, Open MPI library, and Intel MKL math library:
\begin{verbatim}
./configure --prefix=petsc_real_install_path --with-debugging=no 
--with-64-bit-indices=true --with-cc=mpicc --with-cxx=mpicxx 
--with-fc=mpif90 --with-blas-lapack-lib="-Wl,--start-group
${MKLROOT}/lib/intel64/libmkl_intel_lp64.a 
${MKLROOT}/lib/intel64/libmkl_gnu_thread.a 
${MKLROOT}/lib/intel64/libmkl_core.a -Wl,--end-group -lgomp -lpthread -lm -ldl"
CFLAGS="-O2" CXXFLAGS="-O2" FFLAGS="-O2"

./configure --prefix=petsc_complex_install_path --with-debugging=no
--with-64-bit-indices=true --with-cc=mpicc --with-cxx=mpicxx
--with-fc=mpif90 --with-fortran-kernels=true --with-scalar-type=complex
--with-blas-lapack-lib="-Wl,--start-group 
${MKLROOT}/lib/intel64/libmkl_intel_lp64.a ${MKLROOT}/lib/intel64/libmkl_gnu_thread.a
${MKLROOT}/lib/intel64/libmkl_core.a -Wl,--end-group -lgomp -lpthread -lm -ldl"
CFLAGS="-O2" CXXFLAGS="-O2" FFLAGS="-O2"
\end{verbatim}

\item {\bf SLEPc}: The SLEPc library is built on top of PETSc, and it is used in DFT-FE for Gram-Schmidt Orthogonalization. To install SLEPc, first download the current release (3.15.0 or later) tarball from \url{http://slepc.upv.es/download/}, and then follow the installation procedure described in \url{http://slepc.upv.es/documentation/instal.htm}. {\bf Important: } SLEPc installation requires PETSc to be installed first. You also need to create two separate SLEPc installations- one for PETSc installed with \\\verb|--with-scalar-type=real|, and the second for PETSc installed with \verb|--with-scalar-type=complex|. 
	
For your reference you provide here an example installation of SLEPc for real scalar type
\begin{verbatim}
$ export PETSC_DIR=petscReal_install_dir_path
$ unset PETSC_ARCH
$ cd downloaded_slepc_dir
$ ./configure --prefix=slepcReal_install_dir_path
$ make
$ make install
\end{verbatim}


\item {\bf deal.II}: Assuming PETSc and SLEPc are installed, we now briefly discuss the steps to compile and install deal.II library linked with the above dependencies. You need to install two variants of the deal.II library---one variant linked with real scalar type PETSc and SLEPc installations, and the other variant linked with complex scalar type PETSc and SLEPc installations. 

\begin{verbatim}
$ mkdir buildReal
$ cd buildReal
$ cmake -DCMAKE_INSTALL_PREFIX=dealii_petscReal_install_dir_path
        otherCmakeOptions ../deal.II
$ make install
\end{verbatim}
{\bf ``otherCmakeOptions'' include} the following options
\begin{verbatim}
-DCMAKE_C_COMPILER=c_compiler
-DCMAKE_CXX_COMPILER=cxx_compiler
-DCMAKE_Fortran_COMPILER=fortran_compiler
-DMPI_C_COMPILER=mpi_c_compiler_wrapper
-DMPI_CXX_COMPILER=mpi_cxx_compiler_wrapper
-DMPI_Fortran_COMPILER=mpi_fortran_compiler_wrapper
-DCMAKE_CXX_FLAGS=cxx_flags
-DCMAKE_C_FLAGS=c_flags
-DDEAL_II_WITH_MPI=ON -DDEAL_II_WITH_64BIT_INDICES=ON
-DDEAL_II_WITH_P4EST=ON -DP4EST_DIR=p4est_install_dir_path
-DDEAL_II_WITH_PETSC=ON -DPETSC_DIR=petscReal_install_dir_path 
-DDEAL_II_WITH_SLEPC=ON -DSLEPC_DIR=slepcReal_install_dir_path
-DDEAL_II_WITH_LAPACK=ON
-DLAPACK_DIR=lapack_dir_path 
-DLAPACK_FOUND=true 
-DLAPACK_LIBRARIES=lapack_lib_path
-DSCALAPACK_DIR=scalapack_dir_path (only required if linking to Netlib ScaLAPACK)
-DSCALAPACK_LIBRARIES=scalapack_lib_path
-DDEAL_II_WITH_TBB=OFF
-DDEAL_II_WITH_TASKFLOW=OFF
-DDEAL_II_COMPONENT_EXAMPLES=OFF
\end{verbatim}

\item {\bf DFT-FE}: Follow the same instructions as in Sec.~\ref{sec:dftfeinstall}, except to modify the \verb|setupUserPetsc.sh| script instead of the \verb|setupUser.sh|. In \verb|setupUserPetsc.sh|, in addition to updating the paths and flags as discussed in Sec~\ref{sec:dftfeinstall}, update the dealii installation paths from the previous step as follows:
\begin{verbatim}
dealiiPetscRealDir=dealii_petscReal_install_dir_path
dealiiPetscComplexDir=dealii_petscComplex_install_dir_path
\end{verbatim}
\end{enumerate}


\subsection{Important generic instructions}
\begin{itemize}
\item We strongly recommend to link to optimized BLAS-LAPACK library. If using Intel MKL for BLAS-LAPACK library, it is {\bf very important} to use \href{https://software.intel.com/en-us/articles/intel-mkl-link-line-advisor}{Intel MKL Link Line Advisor} to correctly link with Intel MKL for installations of PETSc, ScaLAPACK, ELPA, deal.II, and PETSc. To exploit performance benefit from threads, we recommend (strongly recommended for the new Intel Xeon Phi (KNL) and Skylake processors) linking to threaded versions of Intel MKL libraries by using the options ``threading layer'' and  ``OpenMP library'' in \href{https://software.intel.com/en-us/articles/intel-mkl-link-line-advisor}{Intel MKL Link Line Advisor}.

\item Use \verb|-fPIC| compiler flag for compilation of \dftfe{} and its dependencies, to prevent linking errors during \dftfe{} compilation.	

\item \textcolor{red}{\bf CAUTION! It is  highly recommended to compile deal.II, p4est, ScaLAPACK, ELPA, \dftfe, PETSc and SLEPc~ with the same compilers, same BLAS-LAPACK libraries (if applicable), and same MPI libraries. This prevents deal.II compilation issues, occurrence of run time crashes, and \dftfe{} performance degradation.}  
\end{itemize}


\section{Running \dftfe}
\label{sec:run}
\input{rundftfe}

%\section{Future extensions to \dftfe}
%\label{sec:future}
%Some bullet points. To be expanded upon
\begin{itemize}
\item Adaptive meshing
\item Implement improved mixing strategies to decrease the number of SCF iterations.
\item Implement localization strategies to reduce computational cost for for large domain sizes (>1000 atoms).
\item Enriched FEM to reduce computational cost of all-electron calculations.
\item Post-processing tools.
\end{itemize} 


\section{Finding answers to more questions}
\label{sec:questions-and-answers}
\input{contacts}

\appendix

\section{Run-time input parameters}
\label{sec:parameters}
The underlying description of the input parameters also includes a ``Standard/Advanced/Developer'' label, which signifies whether an input parameter is
a standard one, or an advanced level parameter, or a developer level one only meant for development purposes. The default values of the ``Advanced'' and ``Developer'' labelled parameters are good enough for almost all cases. However, in some cases user may need to use ``Advanced'' labelled parameters. For user convenience,
all input parameters are also indexed at the end of this manual in Section~\ref{sec:runtime-parameter-index-full}.
% now include a file that describes all currently available run-time parameters
\subsection{Global parameters}
\label{parameters:global}


\begin{itemize}
\item {\it Parameter name:} {\tt REPRODUCIBLE OUTPUT}
\phantomsection\label{parameters:REPRODUCIBLE OUTPUT}
\label{parameters:REPRODUCIBLE_20OUTPUT}


\index[prmindex]{REPRODUCIBLE OUTPUT}
\index[prmindexfull]{REPRODUCIBLE OUTPUT}


{\it Default:} false


{\it Description:} [Developer] Limit output to that which is reprodicible, i.e. don't print timing or absolute paths. This parameter is only used for testing purposes.


{\it Possible values:} A boolean value (true or false)
\item {\it Parameter name:} {\tt VERBOSITY}
\phantomsection\label{parameters:VERBOSITY}


\index[prmindex]{VERBOSITY}
\index[prmindexfull]{VERBOSITY}


{\it Default:} 1


{\it Description:} [Standard] Parameter to control verbosity of terminal output. Ranging from 1 for low, 2 for medium (prints eigenvalues and fractional occupancies at the end of each ground-state solve), 3 for high (prints eigenvalues and fractional occupancies at the end of each self-consistent field iteration), and 4 for very high, which is only meant for code development purposes. VERBOSITY=0 is only used for unit testing and shouldn't be used by standard users.


{\it Possible values:} An integer $n$ such that $0\leq n \leq 4$
\item {\it Parameter name:} {\tt WRITE DENSITY}
\phantomsection\label{parameters:WRITE DENSITY}
\label{parameters:WRITE_20DENSITY}


\index[prmindex]{WRITE DENSITY}
\index[prmindexfull]{WRITE DENSITY}


{\it Default:} false


{\it Description:} [Standard] Writes KSDFT ground state (last ground state solve in case of geometry optimization) electron-density solution fields (FEM mesh nodal values) to densityOutput.vtu file for visualization purposes. The electron-density solution field in densityOutput.vtu is named density. In case of spin-polarized calculation, two additional solution fields- density\_0 and density\_1 are also written where 0 and 1 denote the spin indices. Default: false.


{\it Possible values:} A boolean value (true or false)
\item {\it Parameter name:} {\tt WRITE WFC}
\phantomsection\label{parameters:WRITE WFC}
\label{parameters:WRITE_20WFC}


\index[prmindex]{WRITE WFC}
\index[prmindexfull]{WRITE WFC}


{\it Default:} false


{\it Description:} [Standard] Writes KSDFT ground state (last ground state solve in case of geometry optimization) wavefunction solution fields (FEM mesh nodal values) to wfcOutput.vtu file for visualization purposes. The wavefunction solution fields in wfcOutput.vtu are named wfc\_s\_k\_i in case of spin-polarized calculations and wfc\_k\_i otherwise, where s denotes the spin index (0 or 1), k denotes the k point index starting from 0, and i denotes the Kohn-Sham wavefunction index starting from 0. Default: false.


{\it Possible values:} A boolean value (true or false)
\end{itemize}



\subsection{Parameters in section \tt Boundary conditions}
\label{parameters:Boundary_20conditions}

\begin{itemize}
\item {\it Parameter name:} {\tt PERIODIC1}
\phantomsection\label{parameters:Boundary conditions/PERIODIC1}
\label{parameters:Boundary_20conditions/PERIODIC1}


\index[prmindex]{PERIODIC1}
\index[prmindexfull]{Boundary conditions!PERIODIC1}


{\it Default:} false


{\it Description:} [Standard] Periodicity along the first domain bounding vector.


{\it Possible values:} A boolean value (true or false)
\item {\it Parameter name:} {\tt PERIODIC2}
\phantomsection\label{parameters:Boundary conditions/PERIODIC2}
\label{parameters:Boundary_20conditions/PERIODIC2}


\index[prmindex]{PERIODIC2}
\index[prmindexfull]{Boundary conditions!PERIODIC2}


{\it Default:} false


{\it Description:} [Standard] Periodicity along the second domain bounding vector.


{\it Possible values:} A boolean value (true or false)
\item {\it Parameter name:} {\tt PERIODIC3}
\phantomsection\label{parameters:Boundary conditions/PERIODIC3}
\label{parameters:Boundary_20conditions/PERIODIC3}


\index[prmindex]{PERIODIC3}
\index[prmindexfull]{Boundary conditions!PERIODIC3}


{\it Default:} false


{\it Description:} [Standard] Periodicity along the third domain bounding vector.


{\it Possible values:} A boolean value (true or false)
\item {\it Parameter name:} {\tt SELF POTENTIAL RADIUS}
\phantomsection\label{parameters:Boundary conditions/SELF POTENTIAL RADIUS}
\label{parameters:Boundary_20conditions/SELF_20POTENTIAL_20RADIUS}


\index[prmindex]{SELF POTENTIAL RADIUS}
\index[prmindexfull]{Boundary conditions!SELF POTENTIAL RADIUS}


{\it Default:} 0.0


{\it Description:} [Developer] The radius (in a.u) of the ball around an atom on which self-potential of the associated nuclear charge is solved. For the default value of 0.0, the radius value is automatically determined to accomodate the largest radius possible for the given finite element mesh. The default approach works for most problems.


{\it Possible values:} A floating point number $v$ such that $0 \leq v \leq 10$
\end{itemize}

\subsection{Parameters in section \tt Brillouin zone k point sampling options}
\label{parameters:Brillouin_20zone_20k_20point_20sampling_20options}

\begin{itemize}
\item {\it Parameter name:} {\tt USE GROUP SYMMETRY}
\phantomsection\label{parameters:Brillouin zone k point sampling options/USE GROUP SYMMETRY}
\label{parameters:Brillouin_20zone_20k_20point_20sampling_20options/USE_20GROUP_20SYMMETRY}


\index[prmindex]{USE GROUP SYMMETRY}
\index[prmindexfull]{Brillouin zone k point sampling options!USE GROUP SYMMETRY}


{\it Default:} false


{\it Description:} [Standard] Flag to control whether to use point group symmetries. Currently this feature cannot be used if ION FORCE or CELL STRESS input parameters are set to true.


{\it Possible values:} A boolean value (true or false)
\item {\it Parameter name:} {\tt USE TIME REVERSAL SYMMETRY}
\phantomsection\label{parameters:Brillouin zone k point sampling options/USE TIME REVERSAL SYMMETRY}
\label{parameters:Brillouin_20zone_20k_20point_20sampling_20options/USE_20TIME_20REVERSAL_20SYMMETRY}


\index[prmindex]{USE TIME REVERSAL SYMMETRY}
\index[prmindexfull]{Brillouin zone k point sampling options!USE TIME REVERSAL SYMMETRY}


{\it Default:} false


{\it Description:} [Standard] Flag to control usage of time reversal symmetry.


{\it Possible values:} A boolean value (true or false)
\item {\it Parameter name:} {\tt kPOINT RULE FILE}
\phantomsection\label{parameters:Brillouin zone k point sampling options/kPOINT RULE FILE}
\label{parameters:Brillouin_20zone_20k_20point_20sampling_20options/kPOINT_20RULE_20FILE}


\index[prmindex]{kPOINT RULE FILE}
\index[prmindexfull]{Brillouin zone k point sampling options!kPOINT RULE FILE}


{\it Default:} 


{\it Description:} [Developer] File specifying the k-Point quadrature rule to sample Brillouin zone. CAUTION: This option is only used for postprocessing, for example band structure calculation. To set k point rule for DFT solve use the Monkhorst-Pack (MP) grid generation.


{\it Possible values:} Any string
\end{itemize}



\subsection{Parameters in section \tt Brillouin zone k point sampling options/Monkhorst-Pack (MP) grid generation}
\label{parameters:Brillouin_20zone_20k_20point_20sampling_20options/Monkhorst_2dPack_20_28MP_29_20grid_20generation}

\begin{itemize}
\item {\it Parameter name:} {\tt SAMPLING POINTS 1}
\phantomsection\label{parameters:Brillouin zone k point sampling options/Monkhorst_2dPack _28MP_29 grid generation/SAMPLING POINTS 1}
\label{parameters:Brillouin_20zone_20k_20point_20sampling_20options/Monkhorst_2dPack_20_28MP_29_20grid_20generation/SAMPLING_20POINTS_201}


\index[prmindex]{SAMPLING POINTS 1}
\index[prmindexfull]{Brillouin zone k point sampling options!Monkhorst-Pack (MP) grid generation!SAMPLING POINTS 1}


{\it Default:} 1


{\it Description:} [Standard] Number of Monkhorst-Pack grid points to be used along reciprocal latttice vector 1.


{\it Possible values:} An integer $n$ such that $1\leq n \leq 1000$
\item {\it Parameter name:} {\tt SAMPLING POINTS 2}
\phantomsection\label{parameters:Brillouin zone k point sampling options/Monkhorst_2dPack _28MP_29 grid generation/SAMPLING POINTS 2}
\label{parameters:Brillouin_20zone_20k_20point_20sampling_20options/Monkhorst_2dPack_20_28MP_29_20grid_20generation/SAMPLING_20POINTS_202}


\index[prmindex]{SAMPLING POINTS 2}
\index[prmindexfull]{Brillouin zone k point sampling options!Monkhorst-Pack (MP) grid generation!SAMPLING POINTS 2}


{\it Default:} 1


{\it Description:} [Standard] Number of Monkhorst-Pack grid points to be used along reciprocal latttice vector 2.


{\it Possible values:} An integer $n$ such that $1\leq n \leq 1000$
\item {\it Parameter name:} {\tt SAMPLING POINTS 3}
\phantomsection\label{parameters:Brillouin zone k point sampling options/Monkhorst_2dPack _28MP_29 grid generation/SAMPLING POINTS 3}
\label{parameters:Brillouin_20zone_20k_20point_20sampling_20options/Monkhorst_2dPack_20_28MP_29_20grid_20generation/SAMPLING_20POINTS_203}


\index[prmindex]{SAMPLING POINTS 3}
\index[prmindexfull]{Brillouin zone k point sampling options!Monkhorst-Pack (MP) grid generation!SAMPLING POINTS 3}


{\it Default:} 1


{\it Description:} [Standard] Number of Monkhorst-Pack grid points to be used along reciprocal latttice vector 3.


{\it Possible values:} An integer $n$ such that $1\leq n \leq 1000$
\item {\it Parameter name:} {\tt SAMPLING SHIFT 1}
\phantomsection\label{parameters:Brillouin zone k point sampling options/Monkhorst_2dPack _28MP_29 grid generation/SAMPLING SHIFT 1}
\label{parameters:Brillouin_20zone_20k_20point_20sampling_20options/Monkhorst_2dPack_20_28MP_29_20grid_20generation/SAMPLING_20SHIFT_201}


\index[prmindex]{SAMPLING SHIFT 1}
\index[prmindexfull]{Brillouin zone k point sampling options!Monkhorst-Pack (MP) grid generation!SAMPLING SHIFT 1}


{\it Default:} 0


{\it Description:} [Standard] If fractional shifting to be used (0 for no shift, 1 for shift) along reciprocal latttice vector 1.


{\it Possible values:} An integer $n$ such that $0\leq n \leq 1$
\item {\it Parameter name:} {\tt SAMPLING SHIFT 2}
\phantomsection\label{parameters:Brillouin zone k point sampling options/Monkhorst_2dPack _28MP_29 grid generation/SAMPLING SHIFT 2}
\label{parameters:Brillouin_20zone_20k_20point_20sampling_20options/Monkhorst_2dPack_20_28MP_29_20grid_20generation/SAMPLING_20SHIFT_202}


\index[prmindex]{SAMPLING SHIFT 2}
\index[prmindexfull]{Brillouin zone k point sampling options!Monkhorst-Pack (MP) grid generation!SAMPLING SHIFT 2}


{\it Default:} 0


{\it Description:} [Standard] If fractional shifting to be used (0 for no shift, 1 for shift) along reciprocal latttice vector 2.


{\it Possible values:} An integer $n$ such that $0\leq n \leq 1$
\item {\it Parameter name:} {\tt SAMPLING SHIFT 3}
\phantomsection\label{parameters:Brillouin zone k point sampling options/Monkhorst_2dPack _28MP_29 grid generation/SAMPLING SHIFT 3}
\label{parameters:Brillouin_20zone_20k_20point_20sampling_20options/Monkhorst_2dPack_20_28MP_29_20grid_20generation/SAMPLING_20SHIFT_203}


\index[prmindex]{SAMPLING SHIFT 3}
\index[prmindexfull]{Brillouin zone k point sampling options!Monkhorst-Pack (MP) grid generation!SAMPLING SHIFT 3}


{\it Default:} 0


{\it Description:} [Standard] If fractional shifting to be used (0 for no shift, 1 for shift) along reciprocal latttice vector 3.


{\it Possible values:} An integer $n$ such that $0\leq n \leq 1$
\end{itemize}

\subsection{Parameters in section \tt Checkpointing and Restart}
\label{parameters:Checkpointing_20and_20Restart}

\begin{itemize}
\item {\it Parameter name:} {\tt CHK TYPE}
\phantomsection\label{parameters:Checkpointing and Restart/CHK TYPE}
\label{parameters:Checkpointing_20and_20Restart/CHK_20TYPE}


\index[prmindex]{CHK TYPE}
\index[prmindexfull]{Checkpointing and Restart!CHK TYPE}


{\it Default:} 0


{\it Description:} [Standard] Checkpoint type, 0(dont create any checkpoint), 1(create checkpoint for geometry optimization restart if ION OPT or CELL OPT is set to true. Currently, checkpointing and restart framework doesn't work if both ION OPT and CELL OPT are set to true- the code will throw an error if attempted.), 2(create checkpoint for scf restart. Currently, this option cannot be used if geometry optimization is being performed. The code will throw an error if this option is used in conjunction with geometry optimization.)


{\it Possible values:} An integer $n$ such that $0\leq n \leq 2$
\item {\it Parameter name:} {\tt RESTART FROM CHK}
\phantomsection\label{parameters:Checkpointing and Restart/RESTART FROM CHK}
\label{parameters:Checkpointing_20and_20Restart/RESTART_20FROM_20CHK}


\index[prmindex]{RESTART FROM CHK}
\index[prmindexfull]{Checkpointing and Restart!RESTART FROM CHK}


{\it Default:} false


{\it Description:} [Standard] Boolean parameter specifying if the current job reads from a checkpoint. The nature of the restart corresponds to the CHK TYPE parameter. Hence, the checkpoint being read must have been created using the same value of the CHK TYPE parameter. RESTART FROM CHK is always false for CHK TYPE 0.


{\it Possible values:} A boolean value (true or false)
\end{itemize}

\subsection{Parameters in section \tt DFT functional parameters}
\label{parameters:DFT_20functional_20parameters}

\begin{itemize}
\item {\it Parameter name:} {\tt EXCHANGE CORRELATION TYPE}
\phantomsection\label{parameters:DFT functional parameters/EXCHANGE CORRELATION TYPE}
\label{parameters:DFT_20functional_20parameters/EXCHANGE_20CORRELATION_20TYPE}


\index[prmindex]{EXCHANGE CORRELATION TYPE}
\index[prmindexfull]{DFT functional parameters!EXCHANGE CORRELATION TYPE}


{\it Default:} 1


{\it Description:} [Standard] Parameter specifying the type of exchange-correlation to be used: 1(LDA: Perdew Zunger Ceperley Alder correlation with Slater Exchange[PRB. 23, 5048 (1981)]), 2(LDA: Perdew-Wang 92 functional with Slater Exchange [PRB. 45, 13244 (1992)]), 3(LDA: Vosko, Wilk \& Nusair with Slater Exchange[Can. J. Phys. 58, 1200 (1980)]), 4(GGA: Perdew-Burke-Ernzerhof functional [PRL. 77, 3865 (1996)]).


{\it Possible values:} An integer $n$ such that $1\leq n \leq 4$
\item {\it Parameter name:} {\tt PSEUDOPOTENTIAL CALCULATION}
\phantomsection\label{parameters:DFT functional parameters/PSEUDOPOTENTIAL CALCULATION}
\label{parameters:DFT_20functional_20parameters/PSEUDOPOTENTIAL_20CALCULATION}


\index[prmindex]{PSEUDOPOTENTIAL CALCULATION}
\index[prmindexfull]{DFT functional parameters!PSEUDOPOTENTIAL CALCULATION}


{\it Default:} true


{\it Description:} [Standard] Boolean Parameter specifying whether pseudopotential DFT calculation needs to be performed. For all-electron DFT calculation set to false.


{\it Possible values:} A boolean value (true or false)
\item {\it Parameter name:} {\tt PSEUDOPOTENTIAL FILE NAMES LIST}
\phantomsection\label{parameters:DFT functional parameters/PSEUDOPOTENTIAL FILE NAMES LIST}
\label{parameters:DFT_20functional_20parameters/PSEUDOPOTENTIAL_20FILE_20NAMES_20LIST}


\index[prmindex]{PSEUDOPOTENTIAL FILE NAMES LIST}
\index[prmindexfull]{DFT functional parameters!PSEUDOPOTENTIAL FILE NAMES LIST}


{\it Default:} 


{\it Description:} [Standard] Pseudopotential file. This file contains the list of pseudopotential file names in UPF format corresponding to the atoms involved in the calculations. UPF version greater than 2.0 and norm-conserving pseudopotentials in UPF format are only accepted. File format (example for two atoms Mg(z=12), Al(z=13)): 12 filename1.upf(row1), 13 filename2.upf (row2)


{\it Possible values:} Any string
\item {\it Parameter name:} {\tt PSEUDO TESTS FLAG}
\phantomsection\label{parameters:DFT functional parameters/PSEUDO TESTS FLAG}
\label{parameters:DFT_20functional_20parameters/PSEUDO_20TESTS_20FLAG}


\index[prmindex]{PSEUDO TESTS FLAG}
\index[prmindexfull]{DFT functional parameters!PSEUDO TESTS FLAG}


{\it Default:} false


{\it Description:} [Developer] Boolean parameter specifying the explicit path of pseudopotential upf format files used for ctests


{\it Possible values:} A boolean value (true or false)
\item {\it Parameter name:} {\tt SPIN POLARIZATION}
\phantomsection\label{parameters:DFT functional parameters/SPIN POLARIZATION}
\label{parameters:DFT_20functional_20parameters/SPIN_20POLARIZATION}


\index[prmindex]{SPIN POLARIZATION}
\index[prmindexfull]{DFT functional parameters!SPIN POLARIZATION}


{\it Default:} 0


{\it Description:} [Standard] Spin polarization: 0 for no spin polarization and 1 for spin polarization. Default option is 0.


{\it Possible values:} An integer $n$ such that $0\leq n \leq 1$
\item {\it Parameter name:} {\tt START MAGNETIZATION}
\phantomsection\label{parameters:DFT functional parameters/START MAGNETIZATION}
\label{parameters:DFT_20functional_20parameters/START_20MAGNETIZATION}


\index[prmindex]{START MAGNETIZATION}
\index[prmindexfull]{DFT functional parameters!START MAGNETIZATION}


{\it Default:} 0.0


{\it Description:} [Standard] Magnetization to start with (must be between -0.5 and +0.5). Corresponding magnetization per unit cell will be (2 x START MAGNETIZATION x Ne) a.u. , where Ne is the number of electrons in the unit cell 


{\it Possible values:} A floating point number $v$ such that $-0.5 \leq v \leq 0.5$
\end{itemize}

\subsection{Parameters in section \tt Finite element mesh parameters}
\label{parameters:Finite_20element_20mesh_20parameters}

\begin{itemize}
\item {\it Parameter name:} {\tt MESH FILE}
\phantomsection\label{parameters:Finite element mesh parameters/MESH FILE}
\label{parameters:Finite_20element_20mesh_20parameters/MESH_20FILE}


\index[prmindex]{MESH FILE}
\index[prmindexfull]{Finite element mesh parameters!MESH FILE}


{\it Default:} 


{\it Description:} [Developer] External mesh file path. If nothing is given auto mesh generation is performed. The option is only for testing purposes.


{\it Possible values:} Any string
\item {\it Parameter name:} {\tt POLYNOMIAL ORDER}
\phantomsection\label{parameters:Finite element mesh parameters/POLYNOMIAL ORDER}
\label{parameters:Finite_20element_20mesh_20parameters/POLYNOMIAL_20ORDER}


\index[prmindex]{POLYNOMIAL ORDER}
\index[prmindexfull]{Finite element mesh parameters!POLYNOMIAL ORDER}


{\it Default:} 4


{\it Description:} [Standard] The degree of the finite-element interpolating polynomial. Default value is 4. POLYNOMIAL ORDER= 4 or 5 is usually a good choice for most pseudopotential as well as all-electron problems.


{\it Possible values:} An integer $n$ such that $1\leq n \leq 12$
\end{itemize}



\subsection{Parameters in section \tt Finite element mesh parameters/Auto mesh generation parameters}
\label{parameters:Finite_20element_20mesh_20parameters/Auto_20mesh_20generation_20parameters}

\begin{itemize}
\item {\it Parameter name:} {\tt ATOM BALL RADIUS}
\phantomsection\label{parameters:Finite element mesh parameters/Auto mesh generation parameters/ATOM BALL RADIUS}
\label{parameters:Finite_20element_20mesh_20parameters/Auto_20mesh_20generation_20parameters/ATOM_20BALL_20RADIUS}


\index[prmindex]{ATOM BALL RADIUS}
\index[prmindexfull]{Finite element mesh parameters!Auto mesh generation parameters!ATOM BALL RADIUS}


{\it Default:} 2.0


{\it Description:} [Developer] Radius of ball enclosing every atom inside which the mesh size is set close to MESH SIZE AROUND ATOM. The default value of 2.0 is good enough for most cases. On rare cases, where the nonlocal pseudopotential projectors have a compact supportbeyond 2.0, a slightly larger ATOM BALL RADIUS between 2.0 to 2.5 may be required. Standard users do not need to tune this parameter. Units: a.u.


{\it Possible values:} A floating point number $v$ such that $0 \leq v \leq 3$
\item {\it Parameter name:} {\tt BASE MESH SIZE}
\phantomsection\label{parameters:Finite element mesh parameters/Auto mesh generation parameters/BASE MESH SIZE}
\label{parameters:Finite_20element_20mesh_20parameters/Auto_20mesh_20generation_20parameters/BASE_20MESH_20SIZE}


\index[prmindex]{BASE MESH SIZE}
\index[prmindexfull]{Finite element mesh parameters!Auto mesh generation parameters!BASE MESH SIZE}


{\it Default:} 0.0


{\it Description:} [Developer] Mesh size of the base mesh on which refinement is performed. For the default value of 0.0, a heuristically determined base mesh size is used, which is good enough for most cases. Standard users do not need to tune this parameter. Units: a.u.


{\it Possible values:} A floating point number $v$ such that $0 \leq v \leq 20$
\item {\it Parameter name:} {\tt MESH SIZE AROUND ATOM}
\phantomsection\label{parameters:Finite element mesh parameters/Auto mesh generation parameters/MESH SIZE AROUND ATOM}
\label{parameters:Finite_20element_20mesh_20parameters/Auto_20mesh_20generation_20parameters/MESH_20SIZE_20AROUND_20ATOM}


\index[prmindex]{MESH SIZE AROUND ATOM}
\index[prmindexfull]{Finite element mesh parameters!Auto mesh generation parameters!MESH SIZE AROUND ATOM}


{\it Default:} 1.0


{\it Description:} [Standard] Mesh size in a ball of radius ATOM BALL RADIUS around every atom. For pseudopotential calculations, a value between 0.5 to 1.0 is usually a good choice. For all-electron calculations, a value between 0.1 to 0.3 would be a good starting choice. MESH SIZE AROUND ATOM is the only parameter standard users need to tune to achieve the desired accuracy in their results with respect to the mesh refinement. Units: a.u.


{\it Possible values:} A floating point number $v$ such that $0.0001 \leq v \leq 10$
\item {\it Parameter name:} {\tt MESH SIZE AT ATOM}
\phantomsection\label{parameters:Finite element mesh parameters/Auto mesh generation parameters/MESH SIZE AT ATOM}
\label{parameters:Finite_20element_20mesh_20parameters/Auto_20mesh_20generation_20parameters/MESH_20SIZE_20AT_20ATOM}


\index[prmindex]{MESH SIZE AT ATOM}
\index[prmindexfull]{Finite element mesh parameters!Auto mesh generation parameters!MESH SIZE AT ATOM}


{\it Default:} 0.0


{\it Description:} [Developer] Mesh size of the finite elements in the immediate vicinity of the atom. For the default value of 0.0, a heuristically determined MESH SIZE AT ATOM is used, which is good enough for most cases. Standard users do not need to tune this parameter. Units: a.u.


{\it Possible values:} A floating point number $v$ such that $0 \leq v \leq 10$
\end{itemize}

\subsection{Parameters in section \tt Geometry}
\label{parameters:Geometry}

\begin{itemize}
\item {\it Parameter name:} {\tt ATOMIC COORDINATES FILE}
\phantomsection\label{parameters:Geometry/ATOMIC COORDINATES FILE}
\label{parameters:Geometry/ATOMIC_20COORDINATES_20FILE}


\index[prmindex]{ATOMIC COORDINATES FILE}
\index[prmindexfull]{Geometry!ATOMIC COORDINATES FILE}


{\it Default:} 


{\it Description:} [Standard] Atomic-coordinates input file name. For fully non-periodic domain give cartesian coordinates of the atoms (in a.u) with respect to origin at the center of the domain. For periodic and semi-periodic give fractional coordinates of atoms. File format (example for two atoms): Atom1-atomic-charge Atom1-valence-charge x1 y1 z1 (row1), Atom2-atomic-charge Atom2-valence-charge x2 y2 z2 (row2). The number of rows must be equal to NATOMS, and number of unique atoms must be equal to NATOM TYPES.


{\it Possible values:} Any string
\item {\it Parameter name:} {\tt DOMAIN VECTORS FILE}
\phantomsection\label{parameters:Geometry/DOMAIN VECTORS FILE}
\label{parameters:Geometry/DOMAIN_20VECTORS_20FILE}


\index[prmindex]{DOMAIN VECTORS FILE}
\index[prmindexfull]{Geometry!DOMAIN VECTORS FILE}


{\it Default:} 


{\it Description:} [Standard] Domain vectors input file name. Domain vectors describe the edges of the 3D parallelepiped computational domain. File format: v1x v1y v1z (row1), v2x v2y v2z (row2), v3x v3y v3z (row3). Units: a.u. CAUTION: please ensure that the domain vectors form a right-handed coordinate system i.e. dotProduct(crossProduct(v1,v2),v3)>0. Domain vectors are the typical lattice vectors in a fully periodic calculation.


{\it Possible values:} Any string
\item {\it Parameter name:} {\tt NATOMS}
\phantomsection\label{parameters:Geometry/NATOMS}


\index[prmindex]{NATOMS}
\index[prmindexfull]{Geometry!NATOMS}


{\it Default:} 0


{\it Description:} [Standard] Total number of atoms. This parameter requires a mandatory non-zero input which is equal to the number of rows in the file passed to ATOMIC COORDINATES FILE.


{\it Possible values:} An integer $n$ such that $0\leq n \leq 2147483647$
\item {\it Parameter name:} {\tt NATOM TYPES}
\phantomsection\label{parameters:Geometry/NATOM TYPES}
\label{parameters:Geometry/NATOM_20TYPES}


\index[prmindex]{NATOM TYPES}
\index[prmindexfull]{Geometry!NATOM TYPES}


{\it Default:} 0


{\it Description:} [Standard] Total number of atom types. This parameter requires a mandatory non-zero input which is equal to the number of unique atom types in the file passed to ATOMIC COORDINATES FILE.


{\it Possible values:} An integer $n$ such that $0\leq n \leq 2147483647$
\end{itemize}



\subsection{Parameters in section \tt Geometry/Optimization}
\label{parameters:Geometry/Optimization}

\begin{itemize}
\item {\it Parameter name:} {\tt CELL CONSTRAINT TYPE}
\phantomsection\label{parameters:Geometry/Optimization/CELL CONSTRAINT TYPE}
\label{parameters:Geometry/Optimization/CELL_20CONSTRAINT_20TYPE}


\index[prmindex]{CELL CONSTRAINT TYPE}
\index[prmindexfull]{Geometry!Optimization!CELL CONSTRAINT TYPE}


{\it Default:} 12


{\it Description:} [Standard] Cell relaxation constraint type, 1(isotropic shape-fixed volume optimization), 2(volume-fixed shape optimization), 3(relax only cell component v1x), 4(relax only cell component v2x), 5(relax only cell component v3x), 6(relax only cell components v2x and v3x), 7(relax only cell components v1x and v3x), 8(relax only cell components v1x and v2x), 9(volume optimization- relax only v1x, v2x and v3x), 10(2D- relax only x and y components relaxed), 11(2D- relax only x and y shape components- inplane area fixed), 12(relax all cell components), 13 automatically decides the constraints based on boundary conditions. CAUTION: A majority of these options only make sense in an orthorhombic cell geometry.


{\it Possible values:} An integer $n$ such that $1\leq n \leq 13$
\item {\it Parameter name:} {\tt CELL OPT}
\phantomsection\label{parameters:Geometry/Optimization/CELL OPT}
\label{parameters:Geometry/Optimization/CELL_20OPT}


\index[prmindex]{CELL OPT}
\index[prmindexfull]{Geometry!Optimization!CELL OPT}


{\it Default:} false


{\it Description:} [Standard] Boolean parameter specifying if cell stress is to be relaxed


{\it Possible values:} A boolean value (true or false)
\item {\it Parameter name:} {\tt CELL STRESS}
\phantomsection\label{parameters:Geometry/Optimization/CELL STRESS}
\label{parameters:Geometry/Optimization/CELL_20STRESS}


\index[prmindex]{CELL STRESS}
\index[prmindexfull]{Geometry!Optimization!CELL STRESS}


{\it Default:} false


{\it Description:} [Standard] Boolean parameter specifying if cell stress is to be computed. Automatically set to true if CELL OPT is true.


{\it Possible values:} A boolean value (true or false)
\item {\it Parameter name:} {\tt FORCE TOL}
\phantomsection\label{parameters:Geometry/Optimization/FORCE TOL}
\label{parameters:Geometry/Optimization/FORCE_20TOL}


\index[prmindex]{FORCE TOL}
\index[prmindexfull]{Geometry!Optimization!FORCE TOL}


{\it Default:} 1e-4


{\it Description:} [Standard] Sets the tolerance of the maximum force (in a.u.) on an ion when forces are considered to be relaxed.


{\it Possible values:} A floating point number $v$ such that $0 \leq v \leq 1$
\item {\it Parameter name:} {\tt ION FORCE}
\phantomsection\label{parameters:Geometry/Optimization/ION FORCE}
\label{parameters:Geometry/Optimization/ION_20FORCE}


\index[prmindex]{ION FORCE}
\index[prmindexfull]{Geometry!Optimization!ION FORCE}


{\it Default:} false


{\it Description:} [Standard] Boolean parameter specifying if atomic forces are to be computed. Automatically set to true if ION OPT is true.


{\it Possible values:} A boolean value (true or false)
\item {\it Parameter name:} {\tt ION OPT}
\phantomsection\label{parameters:Geometry/Optimization/ION OPT}
\label{parameters:Geometry/Optimization/ION_20OPT}


\index[prmindex]{ION OPT}
\index[prmindexfull]{Geometry!Optimization!ION OPT}


{\it Default:} false


{\it Description:} [Standard] Boolean parameter specifying if atomic forces are to be relaxed.


{\it Possible values:} A boolean value (true or false)
\item {\it Parameter name:} {\tt ION RELAX FLAGS FILE}
\phantomsection\label{parameters:Geometry/Optimization/ION RELAX FLAGS FILE}
\label{parameters:Geometry/Optimization/ION_20RELAX_20FLAGS_20FILE}


\index[prmindex]{ION RELAX FLAGS FILE}
\index[prmindexfull]{Geometry!Optimization!ION RELAX FLAGS FILE}


{\it Default:} 


{\it Description:} [Standard] File specifying the atomic position update permission flags. 1- update 0- no update. File format (example for two atoms with atom 1 fixed and atom 2 free): 0 0 0 (row1), 1 1 1 (row2).


{\it Possible values:} Any string
\item {\it Parameter name:} {\tt NON SELF CONSISTENT FORCE}
\phantomsection\label{parameters:Geometry/Optimization/NON SELF CONSISTENT FORCE}
\label{parameters:Geometry/Optimization/NON_20SELF_20CONSISTENT_20FORCE}


\index[prmindex]{NON SELF CONSISTENT FORCE}
\index[prmindexfull]{Geometry!Optimization!NON SELF CONSISTENT FORCE}


{\it Default:} false


{\it Description:} [Developer] Boolean parameter specfiying whether to add the force terms arising due to the non self-consistency error. Currently non self-consistent force computation is still in developmental phase. The default option is false.


{\it Possible values:} A boolean value (true or false)
\item {\it Parameter name:} {\tt STRESS TOL}
\phantomsection\label{parameters:Geometry/Optimization/STRESS TOL}
\label{parameters:Geometry/Optimization/STRESS_20TOL}


\index[prmindex]{STRESS TOL}
\index[prmindexfull]{Geometry!Optimization!STRESS TOL}


{\it Default:} 1e-6


{\it Description:} [Standard] Sets the tolerance of the cell stress (in a.u.) when cell stress is considered to be relaxed.


{\it Possible values:} A floating point number $v$ such that $0 \leq v \leq 1$
\end{itemize}

\subsection{Parameters in section \tt Parallelization}
\label{parameters:Parallelization}

\begin{itemize}
\item {\it Parameter name:} {\tt NPBAND}
\phantomsection\label{parameters:Parallelization/NPBAND}


\index[prmindex]{NPBAND}
\index[prmindexfull]{Parallelization!NPBAND}


{\it Default:} 1


{\it Description:} [Standard] Number of pools of MPI processors across which the work load of the bands is parallelised. NPKPT times NPBAND must be a divisor of total number of MPI tasks. Further, NPBAND must be less than or equal to NUMBER OF KOHN-SHAM WAVEFUNCTIONS.


{\it Possible values:} An integer $n$ such that $1\leq n \leq 2147483647$
\item {\it Parameter name:} {\tt NPKPT}
\phantomsection\label{parameters:Parallelization/NPKPT}


\index[prmindex]{NPKPT}
\index[prmindexfull]{Parallelization!NPKPT}


{\it Default:} 1


{\it Description:} [Standard] Number of pools of MPI processors across which the work load of the irreducible k-points is parallelised. NPKPT times NPBAND must be a divisor of total number of MPI tasks. Further, NPKPT must be less than or equal to the number of irreducible k-points.


{\it Possible values:} An integer $n$ such that $1\leq n \leq 2147483647$
\end{itemize}

\subsection{Parameters in section \tt Poisson problem parameters}
\label{parameters:Poisson_20problem_20parameters}

\begin{itemize}
\item {\it Parameter name:} {\tt MAXIMUM ITERATIONS}
\phantomsection\label{parameters:Poisson problem parameters/MAXIMUM ITERATIONS}
\label{parameters:Poisson_20problem_20parameters/MAXIMUM_20ITERATIONS}


\index[prmindex]{MAXIMUM ITERATIONS}
\index[prmindexfull]{Poisson problem parameters!MAXIMUM ITERATIONS}


{\it Default:} 5000


{\it Description:} [Developer] Maximum number of iterations to be allowed for Poisson problem convergence.


{\it Possible values:} An integer $n$ such that $0\leq n \leq 20000$
\item {\it Parameter name:} {\tt TOLERANCE}
\phantomsection\label{parameters:Poisson problem parameters/TOLERANCE}
\label{parameters:Poisson_20problem_20parameters/TOLERANCE}


\index[prmindex]{TOLERANCE}
\index[prmindexfull]{Poisson problem parameters!TOLERANCE}


{\it Default:} 1e-12


{\it Description:} [Developer] Relative tolerance as stopping criterion for Poisson problem convergence.


{\it Possible values:} A floating point number $v$ such that $0 \leq v \leq 1$
\end{itemize}

\subsection{Parameters in section \tt SCF parameters}
\label{parameters:SCF_20parameters}

\begin{itemize}
\item {\it Parameter name:} {\tt ANDERSON SCHEME MIXING HISTORY}
\phantomsection\label{parameters:SCF parameters/ANDERSON SCHEME MIXING HISTORY}
\label{parameters:SCF_20parameters/ANDERSON_20SCHEME_20MIXING_20HISTORY}


\index[prmindex]{ANDERSON SCHEME MIXING HISTORY}
\index[prmindexfull]{SCF parameters!ANDERSON SCHEME MIXING HISTORY}


{\it Default:} 10


{\it Description:} [Standard] Number of SCF iteration history to be considered for mixing the electron-density. For metallic systems, typically a mixing history larger than the default value provides better scf convergence.


{\it Possible values:} An integer $n$ such that $1\leq n \leq 1000$
\item {\it Parameter name:} {\tt ANDERSON SCHEME MIXING PARAMETER}
\phantomsection\label{parameters:SCF parameters/ANDERSON SCHEME MIXING PARAMETER}
\label{parameters:SCF_20parameters/ANDERSON_20SCHEME_20MIXING_20PARAMETER}


\index[prmindex]{ANDERSON SCHEME MIXING PARAMETER}
\index[prmindexfull]{SCF parameters!ANDERSON SCHEME MIXING PARAMETER}


{\it Default:} 0.5


{\it Description:} [Standard] Mixing parameter to be used in Anderson scheme.


{\it Possible values:} A floating point number $v$ such that $0 \leq v \leq 1$
\item {\it Parameter name:} {\tt COMPUTE ENERGY EACH ITER}
\phantomsection\label{parameters:SCF parameters/COMPUTE ENERGY EACH ITER}
\label{parameters:SCF_20parameters/COMPUTE_20ENERGY_20EACH_20ITER}


\index[prmindex]{COMPUTE ENERGY EACH ITER}
\index[prmindexfull]{SCF parameters!COMPUTE ENERGY EACH ITER}


{\it Default:} true


{\it Description:} [Developer] Boolean parameter specifying whether to compute the total energy at the end of every scf. Setting it to false can lead to some time savings.


{\it Possible values:} A boolean value (true or false)
\item {\it Parameter name:} {\tt MAXIMUM ITERATIONS}
\phantomsection\label{parameters:SCF parameters/MAXIMUM ITERATIONS}
\label{parameters:SCF_20parameters/MAXIMUM_20ITERATIONS}


\index[prmindex]{MAXIMUM ITERATIONS}
\index[prmindexfull]{SCF parameters!MAXIMUM ITERATIONS}


{\it Default:} 50


{\it Description:} [Standard] Maximum number of iterations to be allowed for SCF convergence


{\it Possible values:} An integer $n$ such that $1\leq n \leq 1000$
\item {\it Parameter name:} {\tt STARTING WFC}
\phantomsection\label{parameters:SCF parameters/STARTING WFC}
\label{parameters:SCF_20parameters/STARTING_20WFC}


\index[prmindex]{STARTING WFC}
\index[prmindexfull]{SCF parameters!STARTING WFC}


{\it Default:} RANDOM


{\it Description:} [Standard] Sets the type of the starting Kohn-Sham wavefunctions guess: Atomic(Superposition of single atom atomic orbitals. Wavefunctions for which atomic orbitals are not available, random wavefunctions are taken. Currently, atomic orbitals data is not available for all atoms.), Random(The starting guess for all wavefunctions are taken to be random). Default: RANDOM.


{\it Possible values:} Any one of ATOMIC, RANDOM
\item {\it Parameter name:} {\tt TEMPERATURE}
\phantomsection\label{parameters:SCF parameters/TEMPERATURE}
\label{parameters:SCF_20parameters/TEMPERATURE}


\index[prmindex]{TEMPERATURE}
\index[prmindexfull]{SCF parameters!TEMPERATURE}


{\it Default:} 500.0


{\it Description:} [Standard] Fermi-Dirac smearing temperature (in Kelvin).


{\it Possible values:} A floating point number $v$ such that $0 \leq v \leq \text{MAX\_DOUBLE}$
\item {\it Parameter name:} {\tt TOLERANCE}
\phantomsection\label{parameters:SCF parameters/TOLERANCE}
\label{parameters:SCF_20parameters/TOLERANCE}


\index[prmindex]{TOLERANCE}
\index[prmindexfull]{SCF parameters!TOLERANCE}


{\it Default:} 1e-06


{\it Description:} [Standard] SCF iterations stopping tolerance in terms of L2 norm of the electron-density difference between two successive iterations. CAUTION: A tolerance close to 1e-7 or lower can detoriate the SCF convergence due to the round-off errors.


{\it Possible values:} A floating point number $v$ such that $1e-12 \leq v \leq 1$
\end{itemize}



\subsection{Parameters in section \tt SCF parameters/Eigen-solver parameters}
\label{parameters:SCF_20parameters/Eigen_2dsolver_20parameters}

\begin{itemize}
\item {\it Parameter name:} {\tt BATCH GEMM}
\phantomsection\label{parameters:SCF parameters/Eigen_2dsolver parameters/BATCH GEMM}
\label{parameters:SCF_20parameters/Eigen_2dsolver_20parameters/BATCH_20GEMM}


\index[prmindex]{BATCH GEMM}
\index[prmindexfull]{SCF parameters!Eigen-solver parameters!BATCH GEMM}


{\it Default:} true


{\it Description:} [Developer] Boolean parameter specifying whether to use gemm batch blas routines to perform matrix-matrix multiplication operations with groups of matrices, processing a number of groups at once using threads instead of the standard serial route. CAUTION: gemm batch blas routines will only be activated if the CHEBYSHEV FILTER BLOCK SIZE is less than 1000, and intel mkl blas library linked with the dealii installation. Default option is true.


{\it Possible values:} A boolean value (true or false)
\item {\it Parameter name:} {\tt CHEBYSHEV FILTER BLOCK SIZE}
\phantomsection\label{parameters:SCF parameters/Eigen_2dsolver parameters/CHEBYSHEV FILTER BLOCK SIZE}
\label{parameters:SCF_20parameters/Eigen_2dsolver_20parameters/CHEBYSHEV_20FILTER_20BLOCK_20SIZE}


\index[prmindex]{CHEBYSHEV FILTER BLOCK SIZE}
\index[prmindexfull]{SCF parameters!Eigen-solver parameters!CHEBYSHEV FILTER BLOCK SIZE}


{\it Default:} 400


{\it Description:} [Developer] The maximum number of wavefunctions which are handled by one call to the Chebyshev filter. This is useful for optimization purposes. The optimum value is dependent on the computing architecture.


{\it Possible values:} An integer $n$ such that $1\leq n \leq 2147483647$
\item {\it Parameter name:} {\tt CHEBYSHEV FILTER TOLERANCE}
\phantomsection\label{parameters:SCF parameters/Eigen_2dsolver parameters/CHEBYSHEV FILTER TOLERANCE}
\label{parameters:SCF_20parameters/Eigen_2dsolver_20parameters/CHEBYSHEV_20FILTER_20TOLERANCE}


\index[prmindex]{CHEBYSHEV FILTER TOLERANCE}
\index[prmindexfull]{SCF parameters!Eigen-solver parameters!CHEBYSHEV FILTER TOLERANCE}


{\it Default:} 1e-02


{\it Description:} [Developer] Parameter specifying the tolerance to which eigenvectors need to computed using chebyshev filtering approach.


{\it Possible values:} A floating point number $v$ such that $1e-10 \leq v \leq \text{MAX\_DOUBLE}$
\item {\it Parameter name:} {\tt CHEBYSHEV POLYNOMIAL DEGREE}
\phantomsection\label{parameters:SCF parameters/Eigen_2dsolver parameters/CHEBYSHEV POLYNOMIAL DEGREE}
\label{parameters:SCF_20parameters/Eigen_2dsolver_20parameters/CHEBYSHEV_20POLYNOMIAL_20DEGREE}


\index[prmindex]{CHEBYSHEV POLYNOMIAL DEGREE}
\index[prmindexfull]{SCF parameters!Eigen-solver parameters!CHEBYSHEV POLYNOMIAL DEGREE}


{\it Default:} 0


{\it Description:} [Developer] Chebyshev polynomial degree to be employed for the Chebyshev filtering subspace iteration procedure to dampen the unwanted spectrum of the Kohn-Sham Hamiltonian. If set to 0, a default value depending on the upper bound of the eigen-spectrum is used.


{\it Possible values:} An integer $n$ such that $0\leq n \leq 2000$
\item {\it Parameter name:} {\tt ENABLE SWITCH TO GS}
\phantomsection\label{parameters:SCF parameters/Eigen_2dsolver parameters/ENABLE SWITCH TO GS}
\label{parameters:SCF_20parameters/Eigen_2dsolver_20parameters/ENABLE_20SWITCH_20TO_20GS}


\index[prmindex]{ENABLE SWITCH TO GS}
\index[prmindexfull]{SCF parameters!Eigen-solver parameters!ENABLE SWITCH TO GS}


{\it Default:} true


{\it Description:} [Developer] Controls automatic switching to Gram-Schimdt orthogonalization if Lowden Orthogonalization or Pseudo-Gram-Schimdt orthogonalization are unstable. Default option is true.


{\it Possible values:} A boolean value (true or false)
\item {\it Parameter name:} {\tt LOWER BOUND UNWANTED FRAC UPPER}
\phantomsection\label{parameters:SCF parameters/Eigen_2dsolver parameters/LOWER BOUND UNWANTED FRAC UPPER}
\label{parameters:SCF_20parameters/Eigen_2dsolver_20parameters/LOWER_20BOUND_20UNWANTED_20FRAC_20UPPER}


\index[prmindex]{LOWER BOUND UNWANTED FRAC UPPER}
\index[prmindexfull]{SCF parameters!Eigen-solver parameters!LOWER BOUND UNWANTED FRAC UPPER}


{\it Default:} 0


{\it Description:} [Developer] The value of the fraction of the upper bound of the unwanted spectrum, the lower bound of the unwanted spectrum will be set. Default value is 0.


{\it Possible values:} A floating point number $v$ such that $0 \leq v \leq 1$
\item {\it Parameter name:} {\tt LOWER BOUND WANTED SPECTRUM}
\phantomsection\label{parameters:SCF parameters/Eigen_2dsolver parameters/LOWER BOUND WANTED SPECTRUM}
\label{parameters:SCF_20parameters/Eigen_2dsolver_20parameters/LOWER_20BOUND_20WANTED_20SPECTRUM}


\index[prmindex]{LOWER BOUND WANTED SPECTRUM}
\index[prmindexfull]{SCF parameters!Eigen-solver parameters!LOWER BOUND WANTED SPECTRUM}


{\it Default:} -10.0


{\it Description:} [Developer] The lower bound of the wanted eigen spectrum. It is only used for the first iteration of the Chebyshev filtered subspace iteration procedure. A rough estimate based on single atom eigen values can be used here. Default value is good enough for most problems.


{\it Possible values:} A floating point number $v$ such that $-\text{MAX\_DOUBLE} \leq v \leq \text{MAX\_DOUBLE}$
\item {\it Parameter name:} {\tt NUMBER OF KOHN-SHAM WAVEFUNCTIONS}
\phantomsection\label{parameters:SCF parameters/Eigen_2dsolver parameters/NUMBER OF KOHN_2dSHAM WAVEFUNCTIONS}
\label{parameters:SCF_20parameters/Eigen_2dsolver_20parameters/NUMBER_20OF_20KOHN_2dSHAM_20WAVEFUNCTIONS}


\index[prmindex]{NUMBER OF KOHN-SHAM WAVEFUNCTIONS}
\index[prmindexfull]{SCF parameters!Eigen-solver parameters!NUMBER OF KOHN-SHAM WAVEFUNCTIONS}


{\it Default:} 10


{\it Description:} [Standard] Number of Kohn-Sham wavefunctions to be computed. For insulators use N/2+(10-20) and for metals use 20 percent more than N/2 (atleast 10 more). N is the total number of electrons. For spin-polarized calculations this parameter denotes the number of Kohn-Sham wavefunctions to be computed for each spin.


{\it Possible values:} An integer $n$ such that $0\leq n \leq 2147483647$
\item {\it Parameter name:} {\tt ORTHOGONALIZATION TYPE}
\phantomsection\label{parameters:SCF parameters/Eigen_2dsolver parameters/ORTHOGONALIZATION TYPE}
\label{parameters:SCF_20parameters/Eigen_2dsolver_20parameters/ORTHOGONALIZATION_20TYPE}


\index[prmindex]{ORTHOGONALIZATION TYPE}
\index[prmindexfull]{SCF parameters!Eigen-solver parameters!ORTHOGONALIZATION TYPE}


{\it Default:} LW


{\it Description:} [Standard] Parameter specifying the type of orthogonalization to be used: GS(Gram-Schmidt Orthogonalization using SLEPc library), LW(Lowden Orthogonalization using LAPACK, extension to ScaLAPACK not implemented yet), PGS(Pseudo-Gram-Schmidt Orthogonalization, if dealii library is compiled with ScaLAPACK, ScaLAPACK functions are used otherwise LAPACK functions are used). LW is the default option.


{\it Possible values:} Any one of GS, LW, PGS
\item {\it Parameter name:} {\tt ORTHO RR WFC BLOCK SIZE}
\phantomsection\label{parameters:SCF parameters/Eigen_2dsolver parameters/ORTHO RR WFC BLOCK SIZE}
\label{parameters:SCF_20parameters/Eigen_2dsolver_20parameters/ORTHO_20RR_20WFC_20BLOCK_20SIZE}


\index[prmindex]{ORTHO RR WFC BLOCK SIZE}
\index[prmindexfull]{SCF parameters!Eigen-solver parameters!ORTHO RR WFC BLOCK SIZE}


{\it Default:} 200


{\it Description:} [Developer] This block size is used for memory optimization purposes in the orthogonalization and Rayleigh-Ritz steps. This optimization is only activated if dealii library is compiled with ScaLAPACK. Default value is 200.


{\it Possible values:} An integer $n$ such that $1\leq n \leq 2147483647$
\item {\it Parameter name:} {\tt SCALAPACKPROCS}
\phantomsection\label{parameters:SCF parameters/Eigen_2dsolver parameters/SCALAPACKPROCS}
\label{parameters:SCF_20parameters/Eigen_2dsolver_20parameters/SCALAPACKPROCS}


\index[prmindex]{SCALAPACKPROCS}
\index[prmindexfull]{SCF parameters!Eigen-solver parameters!SCALAPACKPROCS}


{\it Default:} 0


{\it Description:} [Developer] Uses a processor grid of SCALAPACKPROCS times SCALAPACKPROCS for parallel distribution of the subspace projected matrix in the Rayleigh-Ritz step and the overlap matrix in the Pseudo-Gram-Schmidt step. Default value is 0 for which a thumb rule is used (see http://netlib.org/scalapack/slug/node106.html\#SECTION04511000000000000000). This parameter is only used if dealii library is compiled with ScaLAPACK.


{\it Possible values:} An integer $n$ such that $0\leq n \leq 300$
\item {\it Parameter name:} {\tt SUBSPACE ROT DOFS BLOCK SIZE}
\phantomsection\label{parameters:SCF parameters/Eigen_2dsolver parameters/SUBSPACE ROT DOFS BLOCK SIZE}
\label{parameters:SCF_20parameters/Eigen_2dsolver_20parameters/SUBSPACE_20ROT_20DOFS_20BLOCK_20SIZE}


\index[prmindex]{SUBSPACE ROT DOFS BLOCK SIZE}
\index[prmindexfull]{SCF parameters!Eigen-solver parameters!SUBSPACE ROT DOFS BLOCK SIZE}


{\it Default:} 2000


{\it Description:} [Developer] This block size is used for memory optimization purposes in subspace rotation step in Pseudo-Gram-Schmidt orthogonalization and Rayleigh-Ritz steps. This optimization is only activated if dealii library is compiled with ScaLAPACK. Default value is 2000.


{\it Possible values:} An integer $n$ such that $1\leq n \leq 2147483647$
\end{itemize}



\pagebreak

% print the list of references. make sure the page number in the index is
% correct by putting the \addcontentsline inside the command that prints the
% title of the page, see http://www.dfki.de/~loeckelt/latexbib.html

\let\myRefname\refname
\renewcommand\refname{%
  \addcontentsline{toc}{section}{\numberline{}References}
  \myRefname
}
\bibliographystyle{alpha}
\bibliography{manual}


\pagebreak


\indexprologue{The following is a listing of all run-time parameters, sorted
  by the section in which they appear. 
  \addcontentsline{toc}{section}{\numberline{}Index of run-time parameters with
    section names}
  \label{sec:runtime-parameter-index-full}
}
\printindex[prmindexfull]

\end{document}
