\documentclass{article}
\usepackage[pdftex]{graphicx,color}
\usepackage[T1]{fontenc}
\usepackage{lmodern}
\usepackage{amsmath}
\usepackage{amsfonts}
\usepackage{subcaption}
\usepackage{textpos}

% use a larger page size; otherwise, it is difficult to have complete
% code listings and output on a single page
\usepackage{fullpage}

% have an index. we use the imakeidx' replacement of the 'multind' package so
% that we can have an index of all run-time parameters separate from other
% items (if we ever wanted one)
\usepackage{imakeidx}
\makeindex[name=prmindex, title=Index of run-time parameter entries]
\makeindex[name=prmindexfull, title=Index of run-time parameters with section names]

% be able to use \note environments with a box around the text
\usepackage{fancybox}
\newcommand{\note}[1]{
{\parindent0pt
  \begin{center}
    \shadowbox{
      \begin{minipage}[c]{0.9\linewidth}
        \textbf{Note:} #1
      \end{minipage}
    }
  \end{center}
}}

% use the listings package for code snippets. define keywords for prm files
% and for gnuplot
\usepackage{listings}
\lstset{
  language=C++,
  showstringspaces=false,
  basicstyle=\small\ttfamily,
  columns=fullflexible,
  keepspaces=true,
  frame=single,
  breaklines=true,
  postbreak=\raisebox{0ex}[0ex][0ex]{\hspace{5em}\ensuremath{\color{red}\hookrightarrow\space}}
}
\lstdefinelanguage{prmfile}{morekeywords={set,subsection,end},
                            morecomment=[l]{\#},escapeinside={\%\%}{\%},}
\lstdefinelanguage{gnuplot}{morekeywords={plot,using,title,with,set,replot},
                            morecomment=[l]{\#},}


% use the hyperref package; set the base for relative links to
% the top-level \dftfe directory so that we can link to
% files in the \dftfe tree without having to specify the
% location relative to the directory where the pdf actually
% resides
\usepackage[colorlinks,linkcolor=blue,urlcolor=blue,citecolor=blue,baseurl=../]{hyperref}

\newcommand{\dealii}{{\textsc{deal.II}}}
\newcommand{\pfrst}{{\normalfont\textsc{p4est}}}
\newcommand{\trilinos}{{\textsc{Trilinos}}}
\newcommand{\petsc}{{\textsc{PETSc}}}
\newcommand{\dftfe}{\textsc{DFT-FE}}

\begin{document}

%%%%%%%%%%%%%%%%%%%%%%%%%%%%%%
%%% START OF DFT-FE MANUAL COVER TEMPLATE %%%
%%%%%%%%%%%%%%%%%%%%%%%%%%%%%%
% This should be pasted at the start of manuals and appropriate strings entered at locations indicated with FILL.
% Be sure the TeX file includes the following packages.
% \usepackage{graphicx}
% \usepackage{times}
% \usepackage{textpos}

\definecolor{dark_grey}{gray}{0.3}
\definecolor{dftfe_blue}{rgb}{0.0,0.39,0.76}

%LINE 1%
{
\renewcommand{\familydefault}{\sfdefault}

\pagenumbering{gobble}
%\begin{center}
%\resizebox{\textwidth}{!}{\textcolor{dark_grey}{\fontfamily{\sfdefault}\selectfont
%	\href{http://www-personal.umich.edu/~vikramg/}{COMPUTATIONAL MATERIAL PHYSICS GROUP}
%}}
%\vspace{0.05em}
%\hrule

%LINE 2%
%\color{dark_grey}
%\rule{\textwidth}{2pt}

%LINE 3%
%\color{dark_grey}
% FILL: additional organizations
% e.g.: {\Large Organization 1\\Organization 2}
%{\Large }
%\end{center}

%COLOR AND CODENAME BLOCK%
\begin{center}
\resizebox{\textwidth}{!}{\colorbox
% FILL: color of code name text box
% e.g. blue
{dftfe_blue}{\fontfamily{\rmdefault}\selectfont \textcolor{yellow} {
% FILL: name of the code
% You may want to add \hspace to both sides of the codename to better center it, such as:
% \newcommand{\codename}{\hspace{0.1in}CodeName\hspace{0.1in}}
\hspace{0.1in}\dftfe{}\hspace{0.1in}
}}}
\\[12pt]
{\Large Density Functional Theory calculations with Finite-Elements}
\end{center}

%MAIN PICTURE%
\begin{textblock*}{0in}(0.5in,0.3in)
% FILL: image height
% e.g. height=6.5in
\begin{center}
\vspace{1em}
\includegraphics[scale=0.35]{N2.png}
% FILL: image file name
% e.g. cover_image.png
%{contour_5x5x5.pdf}
%{SiCTriplet0000.png}
\hspace{5em}
\end{center}
\end{textblock*}

%USER MANUAL%
\color{dark_grey}
\vspace{1.0em}
\hfill{\Huge \fontfamily{\sfdefault}\selectfont User Manual \\
\raggedleft \huge \fontfamily{\sfdefault}\selectfont Version
% keep the following line as is so that we can replace this using a script:
1.0.0 %VERSION-INFO%
\\\large(generated \today)\\
\vspace{1.5em}
{\Large Sambit Das\,\\Vikram Gavini\,\\Phani Motamarri\\}
\vspace{1.0em}
\large
\noindent with contributions by: \\
    {\Large Krishnendu Ghosh\\}
\vspace{1.0em}
}
%WEBSITE%
\null
\vspace{17em}

{\noindent
{\fontfamily{\sfdefault}\selectfont \href{https://sites.google.com/umich.edu/dftfe}{website of dftfe}}
}

%\begin{textblock*}{0in}(5.25in,-0.8in)
%\includegraphics[height=0.8in]{CoE-vert.png}
%\end{textblock*}

%LINE%
{\noindent
\color{dark_grey}
\rule{\textwidth}{2pt}
}

}
Copyright (c) 2017-2021 The Regents of the University of Michigan and \hyperref[sec:authors]{DFT-FE authors}.
\pagebreak
\pagenumbering{arabic}

%%%%%%%%%%%%%%%%%%%%%%%%%%%%%%
%%%   END OF DFT-FE MANUAL COVER TEMPLATE    %%%
%%%%%%%%%%%%%%%%%%%%%%%%%%%%%%

\pagebreak

\tableofcontents

\pagebreak

\section{Introduction}
\label{sec:intro}
\dftfe{} is a C++ code for materials modeling from first principles using Kohn-Sham density functional theory.
It is based on adaptive finite-element discretization that handles all-electron and pseudopotential calculations in the 
same framework, and incorporates scalable and efficient solvers for the solution of the Kohn-Sham equations. Importantly, \dftfe{} can handle general geometries and boundary conditions, including periodic, semi-periodic and non-periodic systems. \dftfe{} code builds on top of the deal.II library for everything 
that has to do with finite elements, geometries, meshes, etc., and, through deal.II on p4est for parallel adaptive mesh handling.

\subsection{Authors}
\label{sec:authors}
\dftfe{} is hosted by the \href{http://www-personal.umich.edu/~vikramg/}{Computational Materials Physics
group at University of Michigan}, with Vikram Gavini, Associate Professor of Mechanical Engineering and Materials Science and Engineering, as
the principal investigator broadly overseeing this effort. The code is maintained by a group of principal developers, 
who manage the architecture of the code and the core functionalities. Developers with
significant contributions to core functionalities and code architecture in the past who are 
no longer active principal developers, are listed under principal developers emeriti. 
A subset of the principal developers and mentors are administrators. Finally, all contributors who have
contributed to major parts of the DFT-FE code or sent important fixes and enhancements are listed
under contributors. All the underlying lists are in alphabetical order. 

\paragraph{Principal developers}
\begin{itemize}
	\item Sambit Das (University of Michigan, USA).
	\item Phani Motamarri (University of Michigan, USA).
\end{itemize}

\paragraph{Principal developers emeriti}
\begin{itemize}
	\item Krishnendu Ghosh (University of Michigan, USA).
	\item Shiva Rudraraju (University of Wisconsin Madison, USA).	
\end{itemize}

\paragraph{Contributors}
\begin{itemize}
	\item Sambit Das (University of Michigan, USA).
	\item Denis Davydov (University of Erlangen-Nuremberg, Germany).
	\item Krishnendu Ghosh (University of Michigan, USA).
	\item Phani Motamarri (University of Michigan, USA).
	\item Shiva Rudraraju (University of Wisconsin Madison, USA).
	\item Sukhan Parekh (University of Michigan, USA). 	
\end{itemize}

\paragraph{Mentors}
\begin{itemize}
	\item Vikram Gavini (University of Michigan, USA).
\end{itemize}

\subsection{Acknowledgments}
The development of \dftfe{} open source code relating to pseudopotential calculations has been funded in part by the Department of Energy Software Center for Predictive Integrated Structural Materials Science at University of Michigan, and the Toyota Research Institute. The development of \dftfe{} open source code relating to all-electron calculations has been funded by the Department of Energy Basic Energy Science. The methods and algorithms that have been implemented in \dftfe{} are outputs from research activities over many years that have been supported by the Army Research Office, Air Force Office of Scientific Research, Department of Energy Basic Energy Science, National Science Foundation and Toyota Research Institute. 

\subsection{Referencing \dftfe{}}
Please refer to \href{https://sites.google.com/umich.edu/dftfe/publications}{referencing  \dftfe{}} to properly cite the use of \dftfe{} in your scientific work.  { \color{red} \{link needs to be updated if webpage changes\}}


\section{Useful background information}
\label{sec:background}
We refer to the following articles for a background of the methods and algorithms implemented in \dftfe. \\

\noindent 1. P. Motamarri, M.R. Nowak, K. Leiter , J. Knap, V. Gavini, Higher-order adaptive finite-element methods for Kohn-Sham density functional theory, \emph{J. Comput. Phys.} 253, 308-343 (2013).\\
 
 \noindent 2. P. Motamarri, V. Gavini, Subquadratic-scaling subspace projection method for large-scale Kohn-Sham DFT calculations using spectral finite-element discretization, \emph{Phys. Rev. B} 90, 115127 (2014).\\

\noindent 3. P. Motamarri, V. Gavini,  Configurational forces in electronic structure calculations using Kohn-Sham density functional theory, \emph{Phys. Rev. B} 97 165132 (2018).\\

\noindent In addition, below are some useful references on finite element method and some online resources that provide a background of finite elements and their application to the solution of partial differential equations.\\

\noindent 1. T.J.R. Hughes, The finite element method: linear static and dynamic finite element analysis, Dover Publication, 2000.\\

\noindent 2. K.-J. Bathe, Finite element procedures, Klaus-J\"{u}rgen Bathe, 2014.\\

\noindent 3. The finite element method for problems in physics, online course by Krishna Garikipati. \href{https://www.coursera.org/learn/finite-element-method}{Link}\\

\noindent 4. Online lectures on ``Finite element methods in scientific computing" by Wolfgang Bangerth. \href{http://www.math.colostate.edu/~bangerth/videos.html}{Link}\\ 



\section{Installation}
\label{sec:installation}
All the underlying installation instructions assume a Linux operating system.
\subsection{Compiling and installing external libraries}
\dftfe{} is primarily based on the open source finite element library \href{http://www.dealii.org/}{deal.ii}, through which external dependencies
on \href{http://p4est.org/}{p4est}, \href{https://www.mcs.anl.gov/petsc/}{PETSc}, \href{http://slepc.upv.es/}{SLEPc}, and \href{http://www.netlib.org/scalapack/}{ScaLAPACK}. ScaLAPACK is an optional requirement, but strongly recommended for large problem sizes with 10,000 electrons or more. The other required external libraries, which are
not interfaced via deal.ii are \href{http://www.alglib.net/}{ALGLIB}, \href{http://www.tddft.org/programs/libxc/}{Libxc}, \href{https://atztogo.github.io/spglib/}{spglib}, and \href{http://www.xmlsoft.org/}{Libxml2}. Some of the above libraries (PETSc, SLEPc, ScaLAPACK, and Libxml2) are already installed on most high-performance computers. Below, we give brief installation and/or linking instructions for each of the above libraries.

\subsubsection{Instructions for deal.ii and its dependencies p4est, PETSc, SLEPc, and ScaLAPACK}
First, the installation instructions for the dependencies:
\begin{enumerate}
	\item   {\bf p4est}: This library is used by deal.ii to distribute create and distribute finite-element meshes across multiple processors. Download the latest release tarball of p4est from \url{http://www.p4est.org/}. 

\item {\bf PETSc}: 

\item {\bf SLEPc}:

\item {\bf ScaLAPACK}: Installation of ScaLAPACK will not be required in most cases, as most high-performance computers would have ScaLAPACK already installed in the system. For example 	
\end{enumerate}

Assuming the above dependencies are installed, we now briefly discuss the installation instructions for the deal.ii library.

It is {\bf very important} to ensure that deal.ii and its dependencies (p4est, PETSc, SLEPc, and ScaLAPACK), and the \dftfe{} source code are compiled with the same compilers and MPI libraries, to prevent occurence of crashes and runtime slowdown.  

\subsubsection{Instructions for ALGLIB, Libxc, spglib, and Libxml2}
\begin{enumerate}
	\item   {\bf ALGLIB}: Used by \dftfe{} for spline fitting. Download the latest release of the Alglib free C++ edition from \url{http://www.alglib.net/download.ph}. After downloading and unpacking, go to \verb|cpp/src|, and create a shared library using a C++ compiler. For example, using g++ compiler do
\begin{verbatim}
g++ -c -fPIC *.cpp
g++ *.o -shared -o libAlglib.so
\end{verbatim}
\item {\bf Libxc}: Used by \dftfe{} for exchange-correlation functionals. Download the latest release from \url{http://www.tddft.org/programs/libxc/download/}, and follow the recommended installation procedure (using \verb|./configure|) described in \url{http://www.tddft.org/programs/libxc/installation/}.

\item {\bf spglib}: Used by \dftfe{} to find and handle crystal symmetries. Obtain the source code of the development, and follow the installation procedure described in \url{https://atztogo.github.io/spglib/install.html}.   	

\item {\bf Libxml2}: Libxml2 is used by \dftfe{} to read \verb|.xml| files. Usually, Libxml2 is already installed on most high-performance computers. It is usually installed in the default locations like \verb|/usr/lib64| (library path) and \verb|/usr/include/libxml2| (include path). Libxml2 can also be installed by doing
\begin{verbatim}
git clone git://git.gnome.org/libxml2	
\end{verbatim}
and following the installation instructions in the README.
\end{enumerate}

\subsection{Obtaining and Compiling \dftfe{}}
Assume that you have already installed the above external dependencies, next follow the steps below to obtain and compile \dftfe{}.
\begin{enumerate}
\item Download the source code of the latest release of \dftfe{} from \href{wesbite}{here}. After downloading, unpack the file using the command
\begin{verbatim}
tar -xvzf dftfe-0.5.0.tar.gz
\end{verbatim}

\item   \begin{verbatim}
cd dftfe
\end{verbatim}


\item Make sure the Environment Variables for \href{http://www.cmake.org/}{CMake} (version 2.8.12 or later), C and C++ compilers, and MPI libraries are correctly set to the desired choices. In most high-performance computers the paths for these can be set using \href{http://modules.sourceforge.net/}{Environment Modules}. For example, on one of the high-performance computers we develop and test the DFT-FE code, we use the following commands to set the environment
\begin{verbatim}
module load cmake
module load intel/18.0.1
module load openmpi/3.0.0/intel/18.0.1
\end{verbatim}

\item Set paths to external libraries (deal.ii, ALGLIB, Libxc, spglib, and Libxml2), compiler options, and compiler flags in \verb|setup.sh|, which is a script to compile \dftfe{}. A few example \verb|setup.sh| scripts are provided in the \verb|/helpers| folder for your reference. 

\item Set \verb|optimizedFlag=1| in \verb|setup.sh| and do
\begin{verbatim}
./setup.sh
\end{verbatim} 
which compiles \dftfe{} in the release mode (the fast version). If compilation is successful, a \verb|/build| directory will be created with the following executables:
\begin{verbatim}
/build/release/real/main
/build/release/complex/main
\end{verbatim}
One could also set \verb|optimizedFlag=0| in \verb|setup.sh| and do
\begin{verbatim}
./setup.sh
\end{verbatim}
which will create debug mode executables (useful for debugging purposes but much slower than the release mode executable) in the folders
\begin{verbatim}
/build/debug/real/main
/build/debug/complex/main
\end{verbatim}
\end{enumerate}


\section{Running \dftfe}
\label{sec:run}
The \verb|real/main| executable uses only real data-structures, which are sufficient for fully non-periodic problems, and periodic and semi-periodic problems with only one Brillouin zone sampling point at the origin. The \verb|complex/main| executable uses complex data-structrues which are required for periodic and semi-periodic problems with multiple Brillouin zone sampling points.


%\section{Future extensions to \dftfe}
%\label{sec:future}
%The future versions of DFT-FE will focus on the implementation of the following methodologies.
\begin{itemize}
\item \emph{A-posteriori} mesh adaption techniques to construct optimal meshes with minimal user intervention.
\item Improved electron-density mixing strategies to reduce the number of SCF iterations.
\item Localization technique to reduce computational complexity of DFT-FE for both metallic and insulating systems in the same framework.
\item Enrichment of the finite-element basis using single atom Kohn-Sham wavefunctions to reduce the computational cost of all-electron DFT calculations.
\item GPU support for accelerating the numerical algorithms implemented in DFT-FE.	
\end{itemize} 


\section{Finding answers to more questions}
\label{sec:questions-and-answers}
If you have questions that go beyond this manual, there are a number of
resources:
\begin{itemize}
\item For questions about the source code of \dftfe{}, portability, installation,
  etc., use the \dftfe{} development mailing list at
  \url{dft-fe.users@umich.edu}.

\item \dftfe{} is primarily based on the deal.II library. If you have particular questions
  about deal.II, contact
  the mailing lists described at \url{https://www.dealii.org/mail.html}.

\item If you have specific questions about \dftfe{} that are not suitable
  for public and archived mailing lists, you can contact the
  primary developers and mentors:
  \begin{itemize}
  \item Phani Motamarri: \url{phanim@umich.edu}.
  \item Sambit Das: \url{dsambit@umich.edu}.
  \item Vikram Gavini: \url{vikramg@umich.edu} (Mentor).
  \end{itemize}
\end{itemize}


\appendix

\section{Run-time input parameters}
\label{sec:parameters}
The underlying description of the input parameters also includes a ``Standard/Advanced/Developer'' label, which signifies whether an input parameter is
a standard one, or an advanced level parameter, or a developer level one only meant for development purposes. The default values of the ``Advanced'' and ``Developer'' labelled parameters are good enough for almost all cases. However, in some cases user may need to use ``Advanced'' labelled parameters. For user convenience,
all input parameters are also indexed at the end of this manual in Section~\ref{sec:runtime-parameter-index-full}.
% now include a file that describes all currently available run-time parameters
\subsection{Global parameters}
\label{parameters:global}


\begin{itemize}
\item {\it Parameter name:} {\tt REPRODUCIBLE OUTPUT}
\phantomsection\label{parameters:REPRODUCIBLE OUTPUT}
\label{parameters:REPRODUCIBLE_20OUTPUT}


\index[prmindex]{REPRODUCIBLE OUTPUT}
\index[prmindexfull]{REPRODUCIBLE OUTPUT}
{\it Value:} false


{\it Default:} false


{\it Description:} Limit output to that which is reprodicible, i.e. don't print timing or absolute paths.


{\it Possible values:} A boolean value (true or false)
\item {\it Parameter name:} {\tt VERBOSITY}
\phantomsection\label{parameters:VERBOSITY}


\index[prmindex]{VERBOSITY}
\index[prmindexfull]{VERBOSITY}
{\it Value:} 1


{\it Default:} 1


{\it Description:} Parameter to control verbosity of terminal output. 0 for low, 1 for medium, and 2 for high.


{\it Possible values:} An integer $n$ such that $0\leq n \leq 2$
\end{itemize}



\subsection{Parameters in section \tt Boundary conditions}
\label{parameters:Boundary_20conditions}

\begin{itemize}
\item {\it Parameter name:} {\tt PERIODIC1}
\phantomsection\label{parameters:Boundary conditions/PERIODIC1}
\label{parameters:Boundary_20conditions/PERIODIC1}


\index[prmindex]{PERIODIC1}
\index[prmindexfull]{Boundary conditions!PERIODIC1}
{\it Value:} true


{\it Default:} false


{\it Description:} Periodicity along domain bounding vector, v1.


{\it Possible values:} A boolean value (true or false)
\item {\it Parameter name:} {\tt PERIODIC2}
\phantomsection\label{parameters:Boundary conditions/PERIODIC2}
\label{parameters:Boundary_20conditions/PERIODIC2}


\index[prmindex]{PERIODIC2}
\index[prmindexfull]{Boundary conditions!PERIODIC2}
{\it Value:} true


{\it Default:} false


{\it Description:} Periodicity along domain bounding vector, v2.


{\it Possible values:} A boolean value (true or false)
\item {\it Parameter name:} {\tt PERIODIC3}
\phantomsection\label{parameters:Boundary conditions/PERIODIC3}
\label{parameters:Boundary_20conditions/PERIODIC3}


\index[prmindex]{PERIODIC3}
\index[prmindexfull]{Boundary conditions!PERIODIC3}
{\it Value:} true


{\it Default:} false


{\it Description:} Periodicity along domain bounding vector, v3.


{\it Possible values:} A boolean value (true or false)
\item {\it Parameter name:} {\tt SELF POTENTIAL ATOM BALL RADIUS}
\phantomsection\label{parameters:Boundary conditions/SELF POTENTIAL ATOM BALL RADIUS}
\label{parameters:Boundary_20conditions/SELF_20POTENTIAL_20ATOM_20BALL_20RADIUS}


\index[prmindex]{SELF POTENTIAL ATOM BALL RADIUS}
\index[prmindexfull]{Boundary conditions!SELF POTENTIAL ATOM BALL RADIUS}
{\it Value:} 1.6


{\it Default:} 3.0


{\it Description:} The radius (in a.u) of the ball around an atom on which self-potential of the associated nuclear charge is solved


{\it Possible values:} A floating point number $v$ such that $1.5 \leq v \leq 10$
\end{itemize}

\subsection{Parameters in section \tt Brillouin zone k point sampling options}
\label{parameters:Brillouin_20zone_20k_20point_20sampling_20options}

\begin{itemize}
\item {\it Parameter name:} {\tt NUMBER OF POOLS}
\phantomsection\label{parameters:Brillouin zone k point sampling options/NUMBER OF POOLS}
\label{parameters:Brillouin_20zone_20k_20point_20sampling_20options/NUMBER_20OF_20POOLS}


\index[prmindex]{NUMBER OF POOLS}
\index[prmindexfull]{Brillouin zone k point sampling options!NUMBER OF POOLS}
{\it Value:} 2


{\it Default:} 1


{\it Description:} Number of pools the irreducible k-points to be split on should be a divisor of total number of procs and be less than or equal to the number of irreducible k-points


{\it Possible values:} An integer $n$ such that $1\leq n \leq 2147483647$
\item {\it Parameter name:} {\tt USE GROUP SYMMETRY}
\phantomsection\label{parameters:Brillouin zone k point sampling options/USE GROUP SYMMETRY}
\label{parameters:Brillouin_20zone_20k_20point_20sampling_20options/USE_20GROUP_20SYMMETRY}


\index[prmindex]{USE GROUP SYMMETRY}
\index[prmindexfull]{Brillouin zone k point sampling options!USE GROUP SYMMETRY}
{\it Value:} false


{\it Default:} false


{\it Description:} Flag to control whether to use point group symmetries (set to false for relaxation calculation)


{\it Possible values:} A boolean value (true or false)
\item {\it Parameter name:} {\tt USE TIME REVERSAL SYMMETRY}
\phantomsection\label{parameters:Brillouin zone k point sampling options/USE TIME REVERSAL SYMMETRY}
\label{parameters:Brillouin_20zone_20k_20point_20sampling_20options/USE_20TIME_20REVERSAL_20SYMMETRY}


\index[prmindex]{USE TIME REVERSAL SYMMETRY}
\index[prmindexfull]{Brillouin zone k point sampling options!USE TIME REVERSAL SYMMETRY}
{\it Value:} true


{\it Default:} false


{\it Description:} Flag to control usage of time reversal symmetry 


{\it Possible values:} A boolean value (true or false)
\item {\it Parameter name:} {\tt kPOINT RULE FILE}
\phantomsection\label{parameters:Brillouin zone k point sampling options/kPOINT RULE FILE}
\label{parameters:Brillouin_20zone_20k_20point_20sampling_20options/kPOINT_20RULE_20FILE}


\index[prmindex]{kPOINT RULE FILE}
\index[prmindexfull]{Brillouin zone k point sampling options!kPOINT RULE FILE}
{\it Value:} 


{\it Default:} 


{\it Description:} File specifying the k-Point quadrature rule to sample Brillouin zone. CAUTION: This option is only used for postprocessing, for example band structure calculation. To set k point rule for DFT solve use the Monkhorst-Pack (MP) grid generation.


{\it Possible values:} Any string
\end{itemize}



\subsection{Parameters in section \tt Brillouin zone k point sampling options/Monkhorst-Pack (MP) grid generation}
\label{parameters:Brillouin_20zone_20k_20point_20sampling_20options/Monkhorst_2dPack_20_28MP_29_20grid_20generation}

\begin{itemize}
\item {\it Parameter name:} {\tt SAMPLING POINTS 1}
\phantomsection\label{parameters:Brillouin zone k point sampling options/Monkhorst_2dPack _28MP_29 grid generation/SAMPLING POINTS 1}
\label{parameters:Brillouin_20zone_20k_20point_20sampling_20options/Monkhorst_2dPack_20_28MP_29_20grid_20generation/SAMPLING_20POINTS_201}


\index[prmindex]{SAMPLING POINTS 1}
\index[prmindexfull]{Brillouin zone k point sampling options!Monkhorst-Pack (MP) grid generation!SAMPLING POINTS 1}
{\it Value:} 2


{\it Default:} 2


{\it Description:} Number of Monkhorts-Pack grid points to be used along reciprocal latttice vector 1.


{\it Possible values:} An integer $n$ such that $1\leq n \leq 100$
\item {\it Parameter name:} {\tt SAMPLING POINTS 2}
\phantomsection\label{parameters:Brillouin zone k point sampling options/Monkhorst_2dPack _28MP_29 grid generation/SAMPLING POINTS 2}
\label{parameters:Brillouin_20zone_20k_20point_20sampling_20options/Monkhorst_2dPack_20_28MP_29_20grid_20generation/SAMPLING_20POINTS_202}


\index[prmindex]{SAMPLING POINTS 2}
\index[prmindexfull]{Brillouin zone k point sampling options!Monkhorst-Pack (MP) grid generation!SAMPLING POINTS 2}
{\it Value:} 2


{\it Default:} 2


{\it Description:} Number of Monkhorts-Pack grid points to be used along reciprocal latttice vector 2.


{\it Possible values:} An integer $n$ such that $1\leq n \leq 100$
\item {\it Parameter name:} {\tt SAMPLING POINTS 3}
\phantomsection\label{parameters:Brillouin zone k point sampling options/Monkhorst_2dPack _28MP_29 grid generation/SAMPLING POINTS 3}
\label{parameters:Brillouin_20zone_20k_20point_20sampling_20options/Monkhorst_2dPack_20_28MP_29_20grid_20generation/SAMPLING_20POINTS_203}


\index[prmindex]{SAMPLING POINTS 3}
\index[prmindexfull]{Brillouin zone k point sampling options!Monkhorst-Pack (MP) grid generation!SAMPLING POINTS 3}
{\it Value:} 2


{\it Default:} 2


{\it Description:} Number of Monkhorts-Pack grid points to be used along reciprocal latttice vector 3.


{\it Possible values:} An integer $n$ such that $1\leq n \leq 100$
\item {\it Parameter name:} {\tt SAMPLING SHIFT 1}
\phantomsection\label{parameters:Brillouin zone k point sampling options/Monkhorst_2dPack _28MP_29 grid generation/SAMPLING SHIFT 1}
\label{parameters:Brillouin_20zone_20k_20point_20sampling_20options/Monkhorst_2dPack_20_28MP_29_20grid_20generation/SAMPLING_20SHIFT_201}


\index[prmindex]{SAMPLING SHIFT 1}
\index[prmindexfull]{Brillouin zone k point sampling options!Monkhorst-Pack (MP) grid generation!SAMPLING SHIFT 1}
{\it Value:} 0.25


{\it Default:} 0.0


{\it Description:} Fractional shifting to be used along reciprocal latttice vector 1.


{\it Possible values:} A floating point number $v$ such that $0 \leq v \leq 1$
\item {\it Parameter name:} {\tt SAMPLING SHIFT 2}
\phantomsection\label{parameters:Brillouin zone k point sampling options/Monkhorst_2dPack _28MP_29 grid generation/SAMPLING SHIFT 2}
\label{parameters:Brillouin_20zone_20k_20point_20sampling_20options/Monkhorst_2dPack_20_28MP_29_20grid_20generation/SAMPLING_20SHIFT_202}


\index[prmindex]{SAMPLING SHIFT 2}
\index[prmindexfull]{Brillouin zone k point sampling options!Monkhorst-Pack (MP) grid generation!SAMPLING SHIFT 2}
{\it Value:} 0.25


{\it Default:} 0.0


{\it Description:} Fractional shifting to be used along reciprocal latttice vector 2.


{\it Possible values:} A floating point number $v$ such that $0 \leq v \leq 1$
\item {\it Parameter name:} {\tt SAMPLING SHIFT 3}
\phantomsection\label{parameters:Brillouin zone k point sampling options/Monkhorst_2dPack _28MP_29 grid generation/SAMPLING SHIFT 3}
\label{parameters:Brillouin_20zone_20k_20point_20sampling_20options/Monkhorst_2dPack_20_28MP_29_20grid_20generation/SAMPLING_20SHIFT_203}


\index[prmindex]{SAMPLING SHIFT 3}
\index[prmindexfull]{Brillouin zone k point sampling options!Monkhorst-Pack (MP) grid generation!SAMPLING SHIFT 3}
{\it Value:} 0.25


{\it Default:} 0.0


{\it Description:} Fractional shifting to be used along reciprocal latttice vector 3.


{\it Possible values:} A floating point number $v$ such that $0 \leq v \leq 1$
\end{itemize}

\subsection{Parameters in section \tt Checkpointing and Restart}
\label{parameters:Checkpointing_20and_20Restart}

\begin{itemize}
\item {\it Parameter name:} {\tt CHK TYPE}
\phantomsection\label{parameters:Checkpointing and Restart/CHK TYPE}
\label{parameters:Checkpointing_20and_20Restart/CHK_20TYPE}


\index[prmindex]{CHK TYPE}
\index[prmindexfull]{Checkpointing and Restart!CHK TYPE}
{\it Value:} 2


{\it Default:} 0


{\it Description:} Checkpoint type, 0(dont create any checkpoint), 1(create checkpoint only for ion optimization restart if ION OPT is set to true. This option writes the current atomic coordinates and the cg ion relaxation solver state to checkpoint files. This option assumes CELL OPT is set to false. The checkpoint is created at the end of the last ground state solve.), 2(create checkpoint for scf restart. This option also creates checkpoint for ion optimization restart if ION OPT is set to true.)


{\it Possible values:} An integer $n$ such that $0\leq n \leq 2$
\item {\it Parameter name:} {\tt RESTART FROM CHK}
\phantomsection\label{parameters:Checkpointing and Restart/RESTART FROM CHK}
\label{parameters:Checkpointing_20and_20Restart/RESTART_20FROM_20CHK}


\index[prmindex]{RESTART FROM CHK}
\index[prmindexfull]{Checkpointing and Restart!RESTART FROM CHK}
{\it Value:} false


{\it Default:} false


{\it Description:} Boolean parameter specifying if the current job reads from a checkpoint. The nature of the restart corresponds to the CHK TYPE parameter. Hence, the checkpoint being read must have been created using the same value of the CHK TYPE parameter. RESTART FROM CHK is always false for CHK TYPE 0.


{\it Possible values:} A boolean value (true or false)
\end{itemize}

\subsection{Parameters in section \tt DFT functional related parameters}
\label{parameters:DFT_20functional_20related_20parameters}

\begin{itemize}
\item {\it Parameter name:} {\tt EXCHANGE CORRELATION TYPE}
\phantomsection\label{parameters:DFT functional related parameters/EXCHANGE CORRELATION TYPE}
\label{parameters:DFT_20functional_20related_20parameters/EXCHANGE_20CORRELATION_20TYPE}


\index[prmindex]{EXCHANGE CORRELATION TYPE}
\index[prmindexfull]{DFT functional related parameters!EXCHANGE CORRELATION TYPE}
{\it Value:} 4


{\it Default:} 1


{\it Description:} Parameter specifying the type of exchange-correlation to be used: 1(LDA: Perdew Zunger Ceperley Alder correlation with Slater Exchange[PRB. 23, 5048 (1981)]), 2(LDA: Perdew-Wang 92 functional with Slater Exchange [PRB. 45, 13244 (1992)]), 3(LDA: Vosko, Wilk \& Nusair with Slater Exchange[Can. J. Phys. 58, 1200 (1980)]), 4(GGA: Perdew-Burke-Ernzerhof functional [PRL. 77, 3865 (1996)])


{\it Possible values:} An integer $n$ such that $1\leq n \leq 4$
\item {\it Parameter name:} {\tt PSEUDOPOTENTIAL CALCULATION}
\phantomsection\label{parameters:DFT functional related parameters/PSEUDOPOTENTIAL CALCULATION}
\label{parameters:DFT_20functional_20related_20parameters/PSEUDOPOTENTIAL_20CALCULATION}


\index[prmindex]{PSEUDOPOTENTIAL CALCULATION}
\index[prmindexfull]{DFT functional related parameters!PSEUDOPOTENTIAL CALCULATION}
{\it Value:} true


{\it Default:} true


{\it Description:} Boolean Parameter specifying whether pseudopotential DFT calculation needs to be performed


{\it Possible values:} A boolean value (true or false)
\item {\it Parameter name:} {\tt PSEUDOPOTENTIAL TYPE}
\phantomsection\label{parameters:DFT functional related parameters/PSEUDOPOTENTIAL TYPE}
\label{parameters:DFT_20functional_20related_20parameters/PSEUDOPOTENTIAL_20TYPE}


\index[prmindex]{PSEUDOPOTENTIAL TYPE}
\index[prmindexfull]{DFT functional related parameters!PSEUDOPOTENTIAL TYPE}
{\it Value:} 2


{\it Default:} 1


{\it Description:} Type of nonlocal projector to be used: 1 for KB, 2 for ONCV, default is KB


{\it Possible values:} An integer $n$ such that $1\leq n \leq 2$
\item {\it Parameter name:} {\tt SPIN POLARIZATION}
\phantomsection\label{parameters:DFT functional related parameters/SPIN POLARIZATION}
\label{parameters:DFT_20functional_20related_20parameters/SPIN_20POLARIZATION}


\index[prmindex]{SPIN POLARIZATION}
\index[prmindexfull]{DFT functional related parameters!SPIN POLARIZATION}
{\it Value:} 0


{\it Default:} 0


{\it Description:} Spin polarization: 0 for no spin polarization and 1 for spin polarization


{\it Possible values:} An integer $n$ such that $0\leq n \leq 1$
\item {\it Parameter name:} {\tt START MAGNETIZATION}
\phantomsection\label{parameters:DFT functional related parameters/START MAGNETIZATION}
\label{parameters:DFT_20functional_20related_20parameters/START_20MAGNETIZATION}


\index[prmindex]{START MAGNETIZATION}
\index[prmindexfull]{DFT functional related parameters!START MAGNETIZATION}
{\it Value:} 0.0


{\it Default:} 0.0


{\it Description:} Magnetization to start with (must be between -0.5 and +0.5)


{\it Possible values:} A floating point number $v$ such that $-\text{MAX\_DOUBLE} \leq v \leq \text{MAX\_DOUBLE}$
\end{itemize}

\subsection{Parameters in section \tt Eigen-solver/Chebyshev solver related parameters}
\label{parameters:Eigen_2dsolver_2fChebyshev_20solver_20related_20parameters}

\begin{itemize}
\item {\it Parameter name:} {\tt CHEBYSHEV FILTER PASSES}
\phantomsection\label{parameters:Eigen_2dsolver_2fChebyshev solver related parameters/CHEBYSHEV FILTER PASSES}
\label{parameters:Eigen_2dsolver_2fChebyshev_20solver_20related_20parameters/CHEBYSHEV_20FILTER_20PASSES}


\index[prmindex]{CHEBYSHEV FILTER PASSES}
\index[prmindexfull]{Eigen-solver/Chebyshev solver related parameters!CHEBYSHEV FILTER PASSES}
{\it Value:} 1


{\it Default:} 1


{\it Description:} The number of the Chebyshev filter passes per SCF  (Default value is used when the input parameter is not specified


{\it Possible values:} An integer $n$ such that $1\leq n \leq 20$
\item {\it Parameter name:} {\tt CHEBYSHEV POLYNOMIAL DEGREE}
\phantomsection\label{parameters:Eigen_2dsolver_2fChebyshev solver related parameters/CHEBYSHEV POLYNOMIAL DEGREE}
\label{parameters:Eigen_2dsolver_2fChebyshev_20solver_20related_20parameters/CHEBYSHEV_20POLYNOMIAL_20DEGREE}


\index[prmindex]{CHEBYSHEV POLYNOMIAL DEGREE}
\index[prmindexfull]{Eigen-solver/Chebyshev solver related parameters!CHEBYSHEV POLYNOMIAL DEGREE}
{\it Value:} 0


{\it Default:} 0


{\it Description:} The degree of the Chebyshev polynomial to be employed for filtering out the unwanted spectrum (Default value is used when the input parameter value is 0


{\it Possible values:} An integer $n$ such that $0\leq n \leq 2000$
\item {\it Parameter name:} {\tt LOWER BOUND WANTED SPECTRUM}
\phantomsection\label{parameters:Eigen_2dsolver_2fChebyshev solver related parameters/LOWER BOUND WANTED SPECTRUM}
\label{parameters:Eigen_2dsolver_2fChebyshev_20solver_20related_20parameters/LOWER_20BOUND_20WANTED_20SPECTRUM}


\index[prmindex]{LOWER BOUND WANTED SPECTRUM}
\index[prmindexfull]{Eigen-solver/Chebyshev solver related parameters!LOWER BOUND WANTED SPECTRUM}
{\it Value:} -10.0


{\it Default:} -10.0


{\it Description:} The lower bound of the wanted eigen spectrum


{\it Possible values:} A floating point number $v$ such that $-\text{MAX\_DOUBLE} \leq v \leq \text{MAX\_DOUBLE}$
\item {\it Parameter name:} {\tt NUMBER OF KOHN-SHAM WAVEFUNCTIONS}
\phantomsection\label{parameters:Eigen_2dsolver_2fChebyshev solver related parameters/NUMBER OF KOHN_2dSHAM WAVEFUNCTIONS}
\label{parameters:Eigen_2dsolver_2fChebyshev_20solver_20related_20parameters/NUMBER_20OF_20KOHN_2dSHAM_20WAVEFUNCTIONS}


\index[prmindex]{NUMBER OF KOHN-SHAM WAVEFUNCTIONS}
\index[prmindexfull]{Eigen-solver/Chebyshev solver related parameters!NUMBER OF KOHN-SHAM WAVEFUNCTIONS}
{\it Value:} 20


{\it Default:} 10


{\it Description:} Number of Kohn-Sham wavefunctions to be computed. For insulators use N/2+(10-20) and for metals use 20 percent more than N/2 (atleast 10 more). N is the total number of electrons


{\it Possible values:} An integer $n$ such that $0\leq n \leq 2147483647$
\end{itemize}

\subsection{Parameters in section \tt Finite element mesh parameters}
\label{parameters:Finite_20element_20mesh_20parameters}

\begin{itemize}
\item {\it Parameter name:} {\tt MESH FILE}
\phantomsection\label{parameters:Finite element mesh parameters/MESH FILE}
\label{parameters:Finite_20element_20mesh_20parameters/MESH_20FILE}


\index[prmindex]{MESH FILE}
\index[prmindexfull]{Finite element mesh parameters!MESH FILE}
{\it Value:} 


{\it Default:} 


{\it Description:} External mesh file path. If nothing is given auto mesh generation is performed


{\it Possible values:} Any string
\item {\it Parameter name:} {\tt POLYNOMIAL ORDER}
\phantomsection\label{parameters:Finite element mesh parameters/POLYNOMIAL ORDER}
\label{parameters:Finite_20element_20mesh_20parameters/POLYNOMIAL_20ORDER}


\index[prmindex]{POLYNOMIAL ORDER}
\index[prmindexfull]{Finite element mesh parameters!POLYNOMIAL ORDER}
{\it Value:} 4


{\it Default:} 4


{\it Description:} The degree of the finite-element interpolating polynomial


{\it Possible values:} An integer $n$ such that $1\leq n \leq 12$
\end{itemize}



\subsection{Parameters in section \tt Finite element mesh parameters/Auto mesh generation parameters}
\label{parameters:Finite_20element_20mesh_20parameters/Auto_20mesh_20generation_20parameters}

\begin{itemize}
\item {\it Parameter name:} {\tt ATOM BALL RADIUS}
\phantomsection\label{parameters:Finite element mesh parameters/Auto mesh generation parameters/ATOM BALL RADIUS}
\label{parameters:Finite_20element_20mesh_20parameters/Auto_20mesh_20generation_20parameters/ATOM_20BALL_20RADIUS}


\index[prmindex]{ATOM BALL RADIUS}
\index[prmindexfull]{Finite element mesh parameters!Auto mesh generation parameters!ATOM BALL RADIUS}
{\it Value:} 2.0


{\it Default:} 2.0


{\it Description:} Radius of ball enclosing atom


{\it Possible values:} A floating point number $v$ such that $0 \leq v \leq 10$
\item {\it Parameter name:} {\tt BASE MESH SIZE}
\phantomsection\label{parameters:Finite element mesh parameters/Auto mesh generation parameters/BASE MESH SIZE}
\label{parameters:Finite_20element_20mesh_20parameters/Auto_20mesh_20generation_20parameters/BASE_20MESH_20SIZE}


\index[prmindex]{BASE MESH SIZE}
\index[prmindexfull]{Finite element mesh parameters!Auto mesh generation parameters!BASE MESH SIZE}
{\it Value:} 1.0


{\it Default:} 2.0


{\it Description:} Mesh size of the base mesh on which refinement is performed.


{\it Possible values:} A floating point number $v$ such that $0 \leq v \leq 20$
\item {\it Parameter name:} {\tt MAX REFINEMENT STEPS}
\phantomsection\label{parameters:Finite element mesh parameters/Auto mesh generation parameters/MAX REFINEMENT STEPS}
\label{parameters:Finite_20element_20mesh_20parameters/Auto_20mesh_20generation_20parameters/MAX_20REFINEMENT_20STEPS}


\index[prmindex]{MAX REFINEMENT STEPS}
\index[prmindexfull]{Finite element mesh parameters!Auto mesh generation parameters!MAX REFINEMENT STEPS}
{\it Value:} 10


{\it Default:} 10


{\it Description:} Maximum number of refinement steps to be used. The default value is good enough in most cases.


{\it Possible values:} An integer $n$ such that $1\leq n \leq 10$
\item {\it Parameter name:} {\tt MESH SIZE ATOM BALL}
\phantomsection\label{parameters:Finite element mesh parameters/Auto mesh generation parameters/MESH SIZE ATOM BALL}
\label{parameters:Finite_20element_20mesh_20parameters/Auto_20mesh_20generation_20parameters/MESH_20SIZE_20ATOM_20BALL}


\index[prmindex]{MESH SIZE ATOM BALL}
\index[prmindexfull]{Finite element mesh parameters!Auto mesh generation parameters!MESH SIZE ATOM BALL}
{\it Value:} 0.5


{\it Default:} 0.5


{\it Description:} Mesh size in a ball around atom


{\it Possible values:} A floating point number $v$ such that $0 \leq v \leq 10$
\item {\it Parameter name:} {\tt MESH SIZE NEAR ATOM}
\phantomsection\label{parameters:Finite element mesh parameters/Auto mesh generation parameters/MESH SIZE NEAR ATOM}
\label{parameters:Finite_20element_20mesh_20parameters/Auto_20mesh_20generation_20parameters/MESH_20SIZE_20NEAR_20ATOM}


\index[prmindex]{MESH SIZE NEAR ATOM}
\index[prmindexfull]{Finite element mesh parameters!Auto mesh generation parameters!MESH SIZE NEAR ATOM}
{\it Value:} 0.5


{\it Default:} 0.5


{\it Description:} Mesh size near atom. Useful for all-electron case.


{\it Possible values:} A floating point number $v$ such that $0 \leq v \leq 10$
\end{itemize}

\subsection{Parameters in section \tt Geometry}
\label{parameters:Geometry}

\begin{itemize}
\item {\it Parameter name:} {\tt ATOMIC COORDINATES FILE}
\phantomsection\label{parameters:Geometry/ATOMIC COORDINATES FILE}
\label{parameters:Geometry/ATOMIC_20COORDINATES_20FILE}


\index[prmindex]{ATOMIC COORDINATES FILE}
\index[prmindexfull]{Geometry!ATOMIC COORDINATES FILE}
{\it Value:} coordinates.inp


{\it Default:} 


{\it Description:} Atomic-coordinates file. For fully non-periodic domain give cartesian coordinates of the atoms (in a.u) with respect to origin at the center of the domain. For periodic and semi-periodic give fractional coordinates of atoms. File format (example for two atoms): x1 y1 z1 (row1), x2 y2 z2 (row2).


{\it Possible values:} Any string
\item {\it Parameter name:} {\tt DOMAIN BOUNDING VECTORS FILE}
\phantomsection\label{parameters:Geometry/DOMAIN BOUNDING VECTORS FILE}
\label{parameters:Geometry/DOMAIN_20BOUNDING_20VECTORS_20FILE}


\index[prmindex]{DOMAIN BOUNDING VECTORS FILE}
\index[prmindexfull]{Geometry!DOMAIN BOUNDING VECTORS FILE}
{\it Value:} latticeVectors.inp


{\it Default:} 


{\it Description:} Set file specifying the domain bounding vectors v1, v2 and v3 in a.u. with the following format: v1x v1y v1z (row1), v2x v2y v2z (row2), v3x v3y v3z (row3). Domain bounding vectors are the typical lattice vectors in a fully periodic calculation.


{\it Possible values:} Any string
\end{itemize}



\subsection{Parameters in section \tt Geometry/Optimization}
\label{parameters:Geometry/Optimization}

\begin{itemize}
\item {\it Parameter name:} {\tt CELL CONSTRAINT TYPE}
\phantomsection\label{parameters:Geometry/Optimization/CELL CONSTRAINT TYPE}
\label{parameters:Geometry/Optimization/CELL_20CONSTRAINT_20TYPE}


\index[prmindex]{CELL CONSTRAINT TYPE}
\index[prmindexfull]{Geometry!Optimization!CELL CONSTRAINT TYPE}
{\it Value:} 1


{\it Default:} 12


{\it Description:} Cell relaxation constraint type, 1(isotropic shape-fixed volume optimization), 2(volume-fixed shape optimization), 3(relax only cell component v1x), 4(relax only cell component v2x), 5(relax only cell component v3x), 6(relax only cell components v2x and v3x), 7(relax only cell components v1x and v3x), 8(relax only cell components v1x and v2x), 9(volume optimization- relax only v1x, v2x and v3x), 10(2D- relax only x and y components relaxed), 11(2D- relax only x and y shape components- inplane area fixed), 12(relax all cell components), 13 automatically decides the constraints based boundary conditions. CAUTION: A majority of these options only make sense in an orthorhombic cell geometry.


\pagebreak

% print the list of references. make sure the page number in the index is
% correct by putting the \addcontentsline inside the command that prints the
% title of the page, see http://www.dfki.de/~loeckelt/latexbib.html

\let\myRefname\refname
\renewcommand\refname{%
  \addcontentsline{toc}{section}{\numberline{}References}
  \myRefname
}
\bibliographystyle{alpha}
\bibliography{manual}


\pagebreak


\indexprologue{The following is a listing of all run-time parameters, sorted
  by the section in which they appear. 
  \addcontentsline{toc}{section}{\numberline{}Index of run-time parameters with
    section names}
  \label{sec:runtime-parameter-index-full}
}
\printindex[prmindexfull]

\end{document}
